\section{DataStructure}
\subsection{Fenwick}
\begin{lstlisting}[language=C++]
// [1,n] , init!!
template<class T>
struct Fenwick{
#define lb(x) ((x)&-(x))
    static const int N = 100001;
    int n;T a[N];
    void ini(int _n){ fill_n(a+1,n=_n,0);}
    void Pre(){ for(int i=1,j=i+lb(i);i<=n;++i,j=i+lb(i)) if(j<=n) a[j]+=a[i];}
    void add(int x,T d){ for(;x<=n;x+=lb(x)) a[x]+=d;}
    T sum(int x){ T r=0;for(;x>=1;x^=lb(x)) r+=a[x];return r;}
};
\end{lstlisting}
\subsection{KdTree}
\begin{lstlisting}[language=C++]
// =============== MemoryPool ===============
template<class T>struct Pool{
    T *a,**q; int pa,pq;
    Pool(int MAXN){a=new T[MAXN];q=new T*[MAXN];pa=pq=0;}
//  ~Pool(){delete[] a; delete[] q;}
    void clear(){pa=pq=0;}
    T* NEW(){return pq ? q[--pq] : &a[pa++];}
    void DELETE(T* x){q[pq++]=x; return;}
    int IDX(T*x){return x-a;}
};

typedef long long LL;
const LL INF=0x3f3f3f3f3f3f3f3fLL;
#define REP(i,n) for(int i=0;i<n;++i)

// =============== KD-Tree ===============
#define rangeCalc
template<class T>inline bool updmx(T& A,const T& B){return A<B?A=B,1:0;}
template<class T>inline bool updmn(T& A,const T& B){return A>B?A=B,1:0;}
LL dcmp(const LL& x){return x==0 ? 0 : (x<0 ? -1 : 1);}
//please KDT::clear() before use
namespace KDT{
    const int MAXN=1500010;
    const double SGTA=0.8;// Scapegoat Tree Const
    struct Point{
        int x[2];int i;
        int& operator[](int i){return x[i];}
        const int& operator[](int i)const{return x[i];}
        bool operator<(const Point& B){return x[0]<B.x[0]||(x[0]==B.x[0]&&x[1]<B.x[1]);}
    };
    struct ct{
        int sz;
        int mxD[2],mnD[2];
        ct(){sz=0;}
        explicit ct(const Point& P):sz(1){REP(i,2)mxD[i]=mnD[i]=P[i];}
        bool empty()const{return sz==0;}
    }ct0;
    ct& operator +=(ct& A, const ct& B){
        if(A.empty())return A=B;
        if(!B.empty()){
            A.sz+=B.sz;
            REP(i,2)updmx(A.mxD[i],B.mxD[i]),updmn(A.mnD[i],B.mnD[i]);
        }
        return A;
    }
    struct node{
        node *l,*r;Point v;ct c;
        node(){}
        explicit node(const Point& v):v(v),c(ct(v)){}
    };
    Pool<node>pool(MAXN);
    void clear(){pool.clear();}
    struct cmp{
        int D;cmp(int D):D(D){}
        bool operator()(const Point& A,const Point& B)const{return A[D]<B[D];}
    };
    node* build(int d,Point* A,int l,int r){
        if(l>r) return 0;
        int M=(l+r)>>1;
        nth_element(A+l,A+M,A+r+1,cmp(d));
        node *o=new(pool.NEW())node(A[M]);
        o->l=build(!d,A,l,M-1); o->r=build(!d,A,M+1,r);
        if(o->l)o->c+=o->l->c;
        if(o->r)o->c+=o->r->c;
        return o;
    }
    Point A[MAXN];int AN;
    void eraseAll(node* o){if(!o)return;A[++AN]=o->v;eraseAll(o->l);eraseAll(o->r);pool.DELETE(o);}
    void rebuild(node*& o,int d){AN=0;eraseAll(o);o=build(d,A,1,AN);}
    void insert(node*& rt,const Point& iP){
        node *np=new(pool.NEW())node(iP);
        if(!rt){rt=np;return;}
        static node* sk[MAXN];static int tp;tp=0;
        node *o=rt;
        for(int d=0;;d=!d){
            sk[tp++]=o;
            o->c+=np->c;
            if(iP[d]<o->v[d]){
                if(!o->l){o->l=np;break;}
                o=o->l;
            }else{
                if(!o->r){o->r=np;break;}
                o=o->r;
            }
        }
        sk[tp++]=o;
        if(tp<log(rt->c.sz)/log(1/SGTA))return;
        for(o=sk[--tp];(o->l?o->l->c.sz:0)<SGTA*o->c.sz
                &&(o->r?o->r->c.sz:0)<SGTA*o->c.sz;o=sk[--tp]);
        if(o==rt){rebuild(rt,0);return;}
        node* p=sk[--tp];
        rebuild(p->l==o?p->l:p->r,tp&1);
    }
    LL sqr(int x){return (LL)x*x;}
    Point qP;
    LL getdis(node* o){
        LL res=0;
        REP(d,2){
            if(qP[d]<o->c.mnD[d])res+=sqr(o->c.mnD[d]-qP[d]);
            if(o->c.mxD[d]<qP[d])res+=sqr(qP[d]-o->c.mxD[d]);
        }
        return res;
    }
    LL res; Point resP;
    LL res2; Point resP2;
    void query(int d, node* o){
        LL d0=0; REP(i,2)d0+=sqr(o->v[i]-qP[i]);
        if(d0<res||(d0==res&&o->v<resP))res2=res,resP2=resP,res=d0,resP=o->v;
        else if(d0<res2||(d0==res2&&o->v<resP2))res2=d0,resP2=o->v;
        LL dl=o->l?getdis(o->l):INF;
        LL dr=o->r?getdis(o->r):INF;
        if(dl<dr){
            if(dl<=res2)query(!d,o->l);
            if(dr<=res2)query(!d,o->r);
        }else{
            if(dr<=res2)query(!d,o->r);
            if(dl<=res2)query(!d,o->l);
        }
    }
    Point query(node*& rt, const Point& p){
        qP=p;res=INF;res2=INF;query(0,rt);return resP2;
    }
}
\end{lstlisting}
\subsection{ST}
\begin{lstlisting}[language=C++]
// [0,n)
struct ST{
    static const int N = 101010;
    int a[20][N] , lg[N] , (*f)(int,int);
    void Build(int *v,int n,int (*_f)(int,int)){
        f = _f;rep(i,0,n) a[0][i] = v[i];
        rep(i,2,n+1) lg[i] = lg[i>>1]+1;
        rep(i,1,lg[n]+1) rep(j,0,n-(1<<i)+1) a[i][j]=f(a[i-1][j],a[i-1][j+(1<<i>>1)]);
    }
    int rmq(int x,int y){
        if(x>y) swap(x,y);
        int i=lg[y-x+1];
        return f(a[i][x] , a[i][y+1-(1<<i)]);
    }
};
\end{lstlisting}

\section{Graph}
\subsection{BCC}
\begin{lstlisting}[language=C++]
// key contains the id of edges
// _ starts from 0
namespace BCC{
    const int N = 202020;
    vi key , bcc[N];
    int dfn[N] , low[N] , id[N] , st[N] , _st , _;
    void dfs(int c,int dep,vector<pii> g[]){
        int cc=0;st[_st++]=c;
        dfn[c]=low[c]=dep;
        for(auto e:g[c]){
            int t=e.fi;
            if(!dfn[t]){
                dfs(t,dep+1,g);
                low[c]=min(low[c],low[t]);
                if(low[t]>dfn[c]) key.pb(e.se);
            } else if(dfn[t] != dfn[c] - 1 || cc++)
                low[c] = min(low[c] , dfn[t]);
        }
        if(low[c]==dfn[c]){
            do{id[st[--_st]]=_;}while(st[_st]!=c);
            _++;
        }
    }
    int solve(int n,vector<pii> g[]){
        fill_n(dfn,n,_=0);
        fill_n(low,n,_st=0);
        fill_n(bcc,n,key=vi());
        rep(i,0,n) if(!dfn[i]) dfs(i,1,g);
        rep(i,0,n) for(auto j:g[i]) if(id[i]!=id[j.fi])
            bcc[id[i]].pb(id[j.fi]);
        return _;
    }
};
\end{lstlisting}
\subsection{DCC}
\begin{lstlisting}[language=C++]
// key is cuts
// dcc is edges , i->n+j , i(points) , j(bcc_block)
// be care of isolated point
namespace DCC{
    const int N = 202020;
    vi key , dcc[N];
    int dfn[N] , low[N] , st[N] , _st , _;
    void dfs(int c,int dep,const vi g[]){
        int cc=0,out=1<dep;st[_st++]=c;
        dfn[c]=low[c]=dep;
        for(auto t:g[c])
            if(!dfn[t]){
                dfs(t,dep+1,g);
                low[c]=min(low[c],low[t]);
                if(low[t]>=dfn[c]){
                    if(++out==2) key.pb(c);
                    while(st[--_st]!=t) dcc[st[_st]].pb(_);
                    dcc[c].pb(_);dcc[t].pb(_++);
                }
            } else if(dfn[t] != dfn[c] - 1 || cc++)
                low[c] = min(low[c] , dfn[t]);
    }
    int solve(int n,const vi g[]){// n is size of points
        fill_n(dfn,n,_=0);
        fill_n(low,n,_st=0);
        fill_n(dcc,n,key=vi());
        rep(i,0,n) if(!dfn[i]) dfs(i,1,g);
        rep(i,0,n) if(sz(dcc[i]) == 0) dcc[i].pb(_++);
        return _;
    }
}
\end{lstlisting}
\subsection{Dinic}
\begin{lstlisting}[language=C++]
// [0,n) init!!
template<class T>
struct Dinic{
    const static int N = 10101 , M = N * 10;
    int s , t , n , h[N] , cur[N] , level[N] , q[N] , e , ne[M] , to[M];
    T cap[M] , flow;
    void liu(int u,int v,T w){ to[e] = v;ne[e] = h[u];cap[e] = w;h[u] = e++;}
    void link(int u,int v,T w){ liu(u , v , w);liu(v , u , 0);}
    void ini(int _n = N) { fill(h , h + (n=_n) , -1);e = 0;}
    bool bfs(){
        int L = 0 , R = 0;
        fill(level , level + n , -1);
        level[q[R++] = s] = 0;
        while(L < R && level[t] == -1){
            int c = q[L++];
            for(int k=h[c];~k;k=ne[k])
                if(cap[k] > 0 && level[to[k]] == -1)
                    level[q[R++] = to[k]] = level[c] + 1;
        }
        return ~level[t];
    }
    T dfs(int c,T mx){
        if(c == t) return mx;
        T ret = 0;
        for(int &k = cur[c];~k;k = ne[k]){
            if(level[to[k]] == level[c] + 1 && cap[k] > 0){
                T flow = dfs(to[k] , min(mx , cap[k]));
                ret += flow;cap[k] -= flow , cap[k^1] += flow;mx -= flow;
                if(!mx) return ret;
            }
        }
        level[c] = -1;
        return ret;
    }
    T run(int _s,int _t){
        s = _s , t = _t;
        flow = 0;
        while(bfs()){
            copy(h , h + n , cur);
            flow += dfs(s,~0U>>1);
        }
        return flow;
    }
};
\end{lstlisting}
\subsection{KM}
\begin{lstlisting}[language=C++]
// init!! , id starts from 0
template<class T>
struct KM {
    static const int N = 505;
    static const T inf = ~0U>>2;
    int n, m, left[N], pre[N], used[N];
    T g[N][N], Lx[N], Ly[N], slack[N];
    void ini(int _n, int _m) {
        n = _n , m = _m;
        rep(i,0,n) rep(j,0,m) g[i][j] = -inf;
    }
    void go(int now) {
        rep(i,0,m+1) used[i]=0,slack[i]=inf;
        left[m] = now;int u,v;
        for(u=m;~left[u];u=v){
            used[u] = 1;
            T d = inf;
            rep(i,0,m) if(!used[i]){
                T tmp = Lx[left[u]] + Ly[i] - g[left[u]][i];
                if(tmp < slack[i]) slack[i] = tmp, pre[i] = u;
                if(slack[i] < d) d = slack[v=i];
            }
            rep(i,0,m+1) if(used[i]) Lx[left[i]] -= d , Ly[i] += d;
                else slack[i] -= d;
        }
        for(;u!=m;left[u]=left[pre[u]],u=pre[u]);
    }
    T run() {
        fill_n(Lx,n,0);fill_n(Ly,m,0);
        fill_n(left,m,-1);
        rep(i,0,n) go(i);
        T ans = 0;
        rep(i,0,n) ans += Lx[i];
        rep(i,0,m) ans += Ly[i];
        return -ans;
    }
};
\end{lstlisting}
\subsection{Maxmatch}
\begin{lstlisting}[language=C++]
namespace MaxMatch{
    const int N = 5050;
    int link[N],vis[N];
    int dfs(int c,vi g[]){
        for(auto t : g[c])
            if(!vis[t]){
                vis[t] = true;
                if(link[t]==-1||dfs(link[t],g))
                    return link[t]=c,1;
            }
        return 0;
    }
    int solve(int n,int m,vi g[]){
        fill_n(link,m,-1);
        int ret=0;
        rep(i,0,n){
            memset(vis,0,m*sizeof(int));
            ret += dfs(i,g);
        }
        return ret;
    }
}
\end{lstlisting}
\subsection{MinCostMaxFlow}
\begin{lstlisting}[language=C++]
// [0,n) , init!! , inf modify
template<class U,class V>
struct MCMF{
    static const int N = 204 , M = 101010;
    int h[N] , ing[N] , pre[N] , to[M] , ne[M] , e , s , t , n;
    U cap[M];V dis[N] , cost[M];
    void ini(int _n = N){ fill(h , h + (n=_n) , -1);e = 0;}
    void liu(int u,int v,U c,V w){ to[e] = v;ne[e] = h[u];cap[e] = c;cost[e] = w;h[u] = e++;}
    void link(int u,int v,U c,V w){ liu(u,v,c,w);liu(v,u,0,-w); }
    bool spfa(){
        queue<int> Q;
        fill(dis,dis+n,inf);
        ing[s] = true , dis[s] = 0;
        Q.push(s);
        while(!Q.empty()){
            int c = Q.front();Q.pop();ing[c] = false;
            for(int k=h[c];~k;k=ne[k]){
                int v = to[k];
                if(cap[k] <= 0) continue;
                if(dis[c] + cost[k] < dis[v]){
                    dis[v] = dis[c] + cost[k];
                    pre[v] = k;
                    if(!ing[v]) Q.push(v) , ing[v] = true;
                }
            }
        }
        return dis[t] != inf;
    }
    U flow;V mincost;
    pair<U,V> run(int _s,int _t){
        s = _s , t = _t;
        flow = mincost = 0;
        while(spfa()){
            U pl = inf;int p , k;
            for(p=t;p!=s;p=to[k^1]) pl = min(pl , cap[k=pre[p]]);
            for(p=t;p!=s;p=to[k^1]){
                k = pre[p];
                cap[k] -= pl;
                cap[k^1] += pl;
            }
            mincost += pl * dis[t];
            flow += pl;
        }
        return make_pair(flow , mincost);
    }
};
\end{lstlisting}
\subsection{SCC}
\begin{lstlisting}[language=C++]
// _ starts from 0
namespace SCC{
    const int N = 100050;
    int dfn[N],low[N],id[N],st[N],_st,_,cc;
    void dfs(int c,vi g[]){
        dfn[c]=low[c]=++cc;
        st[_st++]=c;
        for(auto t:g[c])
            if(!dfn[t])
                dfs(t,g),low[c]=min(low[c],low[t]);
            else if(!id[t])
                low[c]=min(low[c],dfn[t]);
        if(low[c]==dfn[c]){
            ++_;
            do{id[st[--_st]]=_;}while(st[_st]!=c);
        }
    }
    vi ng[N];
    int solve(int n,vi g[]){
        fill_n(dfn,n,cc=0);
        fill_n(low,n,_st=0);
        fill_n(id,n,_=0);
        rep(i,0,n) if(!dfn[i]) dfs(i,g);
        rep(i,0,n) --id[i];
        fill_n(ng,_,vi());
        rep(i,0,n) for(auto j:g[i]) if(id[i]!=id[j]) ng[id[i]].pb(id[j]);
        return _;
    }
}
\end{lstlisting}
\subsection{ZKW}
\begin{lstlisting}[language=C++]
// [0,n) , init!! , inf modify
template<class U,class V>
struct ZKW{
    static const int N = 1010 , M = 40404;
    int h[N] , ing[N] , v[N] , to[M] , ne[M] , e , s , t , n;
    U cap[M];V dis[N] , cost[M];
    void ini(int _n = N){ fill(h , h + (n=_n) , -1);e = 0;}
    void liu(int u,int v,U c,V w){ to[e] = v;ne[e] = h[u];cap[e] = c;cost[e] = w;h[u] = e++;}
    void link(int u,int v,U c,V w){ liu(u,v,c,w);liu(v,u,0,-w); }
    void spfa(){
        queue<int> Q;
        fill(dis,dis+n,inf);
        ing[t] = true , dis[t] = 0;
        Q.push(t);
        while(!Q.empty()){
            int c = Q.front();Q.pop();ing[c] = false;
            for(int k=h[c];~k;k=ne[k]){
                int v = to[k];
                if(cap[k^1] <= 0) continue;
                if(dis[c] + cost[k^1] < dis[v]){
                    dis[v] = dis[c] + cost[k^1];
                    if(!ing[v]) Q.push(v) , ing[v] = true;
                }
            }
        }
    }
    U flow;V mincost;
    bool modlable(){
        V Min = inf;
        rep(c,0,n) if(v[c])
            for(int k=h[c];~k;k=ne[k]){
                int t=to[k];
                if(!v[t] && cap[k] > 0)
                    Min = min(Min , dis[t] + cost[k] - dis[c]);
            }
        if(Min == inf) return false;
        rep(i,0,n) if(v[i]) dis[i] += Min;
        return true;
    }
    U dfs(int c,U mx){
        if(c == t) return flow += mx , mincost += mx * dis[s] , mx;
        v[c] = true;U ret = 0;
        for(int k=h[c];~k;k=ne[k]){
            int t = to[k];
            if(!v[t] && cap[k] > 0 && dis[c] - cost[k] == dis[t]){
                U tmp = dfs(t , min(cap[k] , mx - ret));
                cap[k] -= tmp , cap[k^1] += tmp;
                ret += tmp;
                if(ret == mx) return ret;
            }
        }
        return ret;
    }
    pair<U,V> run(int _s,int _t){
        s = _s , t = _t;
        spfa();
        flow = mincost = 0;
        do do memset(v,0,sizeof(v[0])*n);
            while(dfs(s,inf));
        while(modlable());
        return make_pair(flow , mincost);
    }
};
\end{lstlisting}

\section{Math}
\subsection{FFT}
\begin{lstlisting}[language=C++]
// len = 2^x >= max(sz(a),sz(b))*2
namespace FFT{
    void fft(vir *F,int len,int o){
        int j = 0 , k , h;
        rep(i,0,len-1){
            if(i < j) swap(F[i] , F[j]);
            for(k=len;j>=(k>>=1);j&=~k);  j|=k;
        }
        for(h=1;h<len;h<<=1){
            vir wn(cos(pi*o/h),sin(pi*o/h));
            for(j=0;j<len;j+=h<<1){
                vir w(1.);
                for(k=j;k<j+h;++k,w=w*wn){
                    vir b = w*F[k+h];
                    F[k+h] = F[k]-b , F[k] = F[k]+b;
                }
            }
        }
        if(o == -1) rep(i,0,len) F[i] = F[i]/(db)len;
    }
    void mult(vir *a,vir *b,int len){
        fft(a,len,1);fft(b,len,1);
        rep(i,0,len) a[i] = a[i] * b[i];
        fft(a,len,-1);
    }
}
\end{lstlisting}
\subsection{Matrix}
\begin{lstlisting}[language=C++]
typedef vector<vi> MM;
MM operator * (const MM a,const MM b){
    int L = sz(a);MM r(L,vi(L,0));
    rep(i,0,L) rep(j,0,L) rep(k,0,L) r[i][j] += a[i][k] * b[k][j];
    return r;
}
MM operator ^ (MM a,ll t){
    int L = sz(a);MM r(L,vi(L,0));
    rep(i,0,L) r[i][i] = 1;
    for(;t;t>>=1,a=a*a) if(t&1) r=r*a;
    return r;
}
\end{lstlisting}
\subsection{NTT}
\begin{lstlisting}[language=C++]
// init!! , G is root , B is base , len = 2^x
namespace NTT {
    const int N = (1<<19) , P = (479 << 21) + 1, G = 3 , B = 2;
    int w[2][N],rev[N];
    ll Pow(ll x, ll t){ll r=1;for(;t;t>>=1,x=x*x%P)if(t&1)r=r*x%P;return r;}
    void ini(){
        ll t = Pow(G,(P-1)/N);
        w[0][0] = w[1][0] = 1;
        rep(i,1,N) w[0][i] = t*w[0][i-1]%P;
        rep(i,1,N) w[1][i] = w[0][N-i];
        rep(i,0,N) for(int j=1;j<N;j*=B) (rev[i]<<=1)|=(i/j)%B;
    }
    void ntt(int *a,int n,int o){
        int tt = N/n;
        rep(i,0,n){
            int j = rev[i]/tt;
            if(i<j) swap(a[i],a[j]);
        }
        for(int i=1;i<n;i<<=1)
            for(int j=0,t=N/(i+i);j<n;j+=i+i)
                for(int k=j,l=0;k<j+i;++k,l+=t){
                    int b = (ll)a[k+i]*w[o][l]%P;
                    a[k+i] = a[k]-b; if(a[k+i]<0) a[k+i]+=P;
                    a[k] = a[k]+b; if(a[k]>=P) a[k]-=P;
                }
        if(o == 1){
            ll inv = Pow(n,P-2);
            rep(i,0,n) a[i] = a[i]*inv%P;
        }
    }
    void mult(int *a,int *b,int len){
        ntt(a,len,0);ntt(b,len,0);
        rep(i,0,len) a[i] = (ll)a[i]*b[i]%P;
        ntt(a,len,1);
    }
}
\end{lstlisting}
\subsection{NTT3}
\begin{lstlisting}[language=C++]
// init!! , G is root , B is base , len = 3^x
namespace NTT3 {
    const int H = 17 , N = 531441 , P = 258280327 , G = 5 , B = 3;
    int w[2][N] , rev[N];
    void ini(){
        ll t = Pow(G,(P-1)/N);
        w[0][0] = w[1][0] = 1;
        rep(i,1,N) w[0][i] = w[0][i-1]*t%P;
        rep(i,1,N) w[1][i] = w[0][N-i];
        rep(i,0,N) for(int j=1;j<N;j*=B) (rev[i]*=B)+=i/j%B;
    }
    void ntt(int *a,int n,int o){ // N -> max array size
        int tt = N/n , d = N/B;
        rep(i,0,n){
            int j = rev[i]/tt;
            if(i<j) swap(a[i],a[j]);
        }
        for(int i=1;i<n;i*=B)
            for(int j=0,t=N/(i*B);j<n;j+=i*B)
                for(int k=0,l=0;k<i;++k,l+=t){
                    int x = a[j+k] , y = a[j+k+i] , z = a[j+k+i+i];
                    a[j+k] = (x + (ll)y*w[o][l] + (ll)z*w[o][l+l])%P;
                    a[j+k+i] = (x + (ll)y*w[o][l+d] + (ll)z*w[o][(l+l+d+d)%N])%P;
                    a[j+k+i+i] = (x + (ll)y*w[o][l+d+d] + (ll)z*w[o][(l+d+d)*2-N])%P;
                }
        if(o == 1){
            ll inv = Pow(n,P-2);
            rep(i,0,n) a[i] = a[i]*inv%P;
        }
    }
    void mult(int *a,int *b,int len){
        ntt(a,len,0);ntt(b,len,0);
        rep(i,0,len) a[i] = (ll)a[i]*b[i]%P;
        ntt(a,len,1);
    }
}
\end{lstlisting}
\subsection{Prepare}
\begin{lstlisting}[language=C++]
//InverseElement
// mod is a prime
int v[N];
rep(i,1,N) v[i] = Pow(i , mod - 2);
// O(nlgn)
v[1] = Pow(1,mod-2);
rep(i,2,N){
    int t = mod/i , k = (ll)v[mod-t*i] * t % mod;
    v[i] = (ll)i * k % mod * k % mod;
}
// O(n)
int pri[N] , _p , vis[N];
v[1] = Pow(1,mod-2);
rep(i,2,N) v[i] = 1;
rep(i,2,N){
    if(!vis[i]) pri[_p++] = i , v[i] = Pow(i,mod-2);
    for(int j=0,o=0;j<_p&&(o=pri[j]*i)<N;++j){
        vis[o] = true;
        v[o] = (ll)v[pri[j]] * v[i] % mod;
        if(pri[j]%i == 0) break;
    }
}
// O(n)

// Mobius
int mu[N];
// all
mu[1] = 1;
rep(i,1,N) for(int j=i+i;j<N;j+=i) mu[j] -= mu[i];
// O(nlgn)
int pri[N] , _p , vis[N];
mu[1] = 1;
rep(i,2,N){
    if(!vis[i]) pri[_p++] = i , mu[i] = -1;
    for(int j=0,o=0;j<_p&&(o=pri[j]*i)<N;++j){
        vis[o] = true;
        if(i%pri[j]) mu[o] = -mu[i];
        else{ mu[o] = 0; break; }
    }
}
// O(n)

// Phi
int pri[N] , _p , vis[N] , phi[N];
rep(i,2,N){
    if(!vis[i]) pri[_p++] = i , phi[i] = i-1;
    for(int j=0,o;j<_p&&(o=pri[j]*i)<N;++j){
        vis[o] = true;
        if(i%pri[j]) phi[o] = (pri[j] - 1) * phi[i];
        else{
            phi[o] = pri[j] * phi[i];
            break;
        }
    }
}
// O(n)
\end{lstlisting}

\section{Others}
\subsection{expression\_parse}
\begin{lstlisting}[language=C++]
#define ass(x) {if(!(x)) throw 1;}
string t;
string::iterator p;
using val = int;
int pri(char ch){
    if(ch == '*') return 3;
    if(ch == '+') return 2;
    if(ch == '-') return 2;
    if(ch == '$') return -1;
}
val cal(char ch,val a,val b){
    if(ch == '+') return a + b;
}

val next();
val expr(char pre,val x){
    if(pri(pre) < pri(*p)){
        auto op = *p++;
        return expr(pre , cal(op , x , expr(op , next())));
    }
    else return x;
}
val next(){
    if(*p == '('){
        p++;
        val res = expr('(' , next());
        ass(*p++ == ')');
        return res;
    } else if(isdigit(*p)){
        val x = 0;
        while(isdigit(*p)){
            ass(*p == '0' || *p == '1');
            x = x * 2 + *p++ - '0';
        }
        return x;
    } else if(*p == '-'){
        p++;
        return -next();
    }
}

bool check(){ // s$
    try{
        if(count(all(t) , '=') != 1) throw 1;
        p = t.begin();
        int res = expr('$' , next());
        return res == 1;
    } catch(...){
        return 0;
    }
}
\end{lstlisting}
\subsection{head}
\begin{lstlisting}[language=C++]
#include <algorithm>
#include <iostream>
#include <cstring>
#include <vector>
#include <cstdio>
#include <string>
#include <cmath>
#include <set>
#include <map>
using namespace std;
typedef long long ll;
typedef double db;
typedef pair<int,int> pii;
typedef vector<int> vi;
#define de(x) cout << #x << "=" << x << endl
#define rep(i,a,b) for(int i=a;i<b;++i)
#define all(x) (x).begin(),(x).end()
#define sz(x) (int)(x).size()
#define mp make_pair
#define pb push_back
#define fi first
#define se second

int main(){

    return 0;
}
\end{lstlisting}

\section{String}
\subsection{ACAutomaton}
\begin{lstlisting}[language=C++]
// [0,L) , N-1 is virtual , 1 is rt , init!!
struct Trie{
    static const int N = 101010 , M = 26;
    int ne[N][M] , fail[N] , fa[N] , rt , L;
    void ini(){ fill_n(ne[fail[0] = N-1],M,0);L = 0;rt = newnode();}
    int newnode(){ fill_n(ne[L],M,0); return L++; }
    void add(char *s){
        int p = rt;
        for(int i=0;s[i];++i){
            int c = s[i] - 'a';// modify
            if(!ne[p][c]) ne[p][c] = newnode() , fa[L-1] = p;
            p = ne[p][c];
        }
    }
    void Build(){
        vi v;v.pb(rt);
        rep(i,0,sz(v)){
            int c = v[i];
            rep(i,0,M) ne[c][i] ?
                v.pb(ne[c][i]) , fail[ne[c][i]] = ne[fail[c]][i] :
                ne[c][i] = ne[fail[c]][i];
        }
    }
};
\end{lstlisting}
\subsection{DoublingArray}
\begin{lstlisting}[language=C++]
namespace Doubling{
    static const int N = 101010;
    int t[N] , wa[N] , wb[N] , sa[N] , h[N];
    void sort(int *x,int *y,int n,int m){
        rep(i,0,m) t[i] = 0;
        rep(i,0,n) t[x[y[i]]]++;
        rep(i,1,m) t[i] += t[i-1];
        per(i,0,n) sa[--t[x[y[i]]]] = y[i];
    }
    bool cmp(int *x,int a,int b,int d){
        return x[a] == x[b] && x[a+d] == x[b+d];
    }
    void da(int *s,int n,int m){
        int *x=wa,*y=wb;
        rep(i,0,n) x[i] = s[i] , y[i] = i;
        sort(x , y , n , m);
        for(int j=1,p=1;p<n;m=p,j<<=1){
            p = 0;rep(i,n-j,n) y[p++] = i;
            rep(i,0,n) if(sa[i] >= j) y[p++] = sa[i] - j;
            sort(x , y , n , m);
            swap(x , y);p = 1;x[sa[0]] = 0;
            rep(i,1,n) x[sa[i]] = cmp(y,sa[i],sa[i-1],j)?p-1:p++;
        }
    }
    void cal_h(int *s,int n,int *rk){
        int j,k=0;
        for(int i=1;i<=n;++i) rk[sa[i]] = i;
        for(int i=0;i<n;h[rk[i++]] = k)
            for(k&&--k,j=sa[rk[i]-1];s[i+k]==s[j+k];++k);
    }
}
struct DA{ // [0,n] , in[n] = 0 , n load
    static const int N = 101010;
    int p[18][N] , rk[N] , in[N] , Log[N] , n;
    void Build(){
        Doubling::da(in,n+1,300);
        Doubling::cal_h(in,n,rk);
        Log[0] = -1;for(int i=1;i<=n;++i) Log[i] = Log[i-1] + (i==(i&(-i)));
        for(int i=1;i<=n;++i) p[0][i] = Doubling::h[i];
        for(int j=1;1<<j<=n;++j){
            int lim = n+1-(1<<j);
            for(int i=1;i<=lim;++i)
                p[j][i] = min(p[j-1][i] , p[j-1][i+(1<<j>>1)]);
        }
    }
    int lcp(int a,int b){
        a = rk[a] , b = rk[b];
        if(a > b) swap(a , b);++a;
        int t = Log[b-a+1];
        return min(p[t][a] , p[t][b-(1<<t)+1]);
    }
};
\end{lstlisting}
\subsection{Exkmp}
\begin{lstlisting}[language=C++]
void exkmp(char *s,int *z,char *t,int *p){
    int lens = strlen(s);
    int lent = strlen(t);
    p[0]=0;
    for(int i=0,x=0,y=0;i<lens;++i){
        if(i<=y) z[i] = min(y-i,p[i-x]);
        while(i + z[i] < lens && z[i] < lent && s[i + z[i]] == t[z[i]]) ++z[i];
        if(y <= i + z[i]) x = i, y = i + z[i];
    }
}

void Exkmp(){
    scanf("%s%s",s,t);
    exkmp(t+1,nt+1,t,nt);
    exkmp(s,ns,t,nt);
}
\end{lstlisting}
\subsection{Kmp}
\begin{lstlisting}[language=C++]
void kmp(char *s,int *ns,char *t,int *nt){
    int lens = strlen(s);
    int lent = strlen(t);
    nt[0] = -1;
    for(int i=0,j=-1;i<lens;++i){
        while(j >= 0 && s[i] != t[j + 1]) j = nt[j];
        if(s[i] == t[j + 1]) ++j;
        ns[i] = j;
        if(j + 1 == lent) j = nt[j];
    }
}
void KMP(){
    scanf("%s%s",s,t);
    kmp(t+1,nt+1,t,nt);
    kmp(s,ns,t,nt);
}
\end{lstlisting}
\subsection{Manacher}
\begin{lstlisting}[language=C++]
// length of pa is two size of str
// pa[i<<1] : odd string
// pa[i<<1|1] : even string
void Manacher(char *s,int n,int *pa){
    pa[0] = 1;
    for(int i=1,j=0;i<(n<<1)-1;++i){
        int p = i >> 1 , q = i - p , r = ((j + 1)>>1) + pa[j] - 1;
        pa[i] = r < q ? 0 : min(r - q + 1 , pa[(j<<1) - i]);
        while(0 <= p - pa[i] && q + pa[i] < n && s[p - pa[i]] == s[q + pa[i]])
            pa[i]++;
        if(q + pa[i] - 1 > r) j = i;
    }
}
\end{lstlisting}
\subsection{PalindromicTree}
\begin{lstlisting}[language=C++]
// [0,p) , 0(even) and 1(odd) is virtual , init!!
struct Palindromic_Tree {
    static const int N = 101010 , M = 26;
    int ne[N][M] , fail[N] , len[N] , S[N] , last , n , p;
    int newnode(int l){
        fill(ne[p] , ne[p] + M , 0);
        len[p] = l;
        return p++;
    }
    void ini(){
        p = 0;newnode(0);newnode(-1);
        S[n = last = 0] = -1;
        fail[0] = 1;
    }
    int get_fail(int x){
        while(S[n - len[x] - 1] != S[n]) x = fail[x];
        return x;
    }
    void add(int c){
        S[++n] = c;
        int cur = get_fail(last);
        if(!ne[cur][c]){
            int now = newnode(len[cur] + 2);
            fail[now] = ne[get_fail(fail[cur])][c];
            ne[cur][c] = now;
        }
        last = ne[cur][c];
    }
};
\end{lstlisting}
\subsection{SuffixAutomaton}
\begin{lstlisting}[language=C++]
// [0,L] , 0 is virtual , 1 is rt , init!!
struct SAM{
    static const int N = 101010 , M = 26;
    int par[N] , l[N] , ne[N][M];
    int rt , last , L;
    void add(int c){
        int p = last , np = ++L;
        fill(ne[np] , ne[np] + M , 0);
        l[np] = l[p] + 1;
        last = np;
        while(p && !ne[p][c]) ne[p][c] = np , p = par[p];
        if(!p) par[np] = rt;
        else{
            int q = ne[p][c];
            if(l[q] == l[p] + 1) par[np] = q;
            else{
                int nq = ++L;
                l[nq] = l[p] + 1;
                copy(ne[q] , ne[q] + M , ne[nq]);
                par[nq] = par[q];
                par[q] = par[np] = nq;
                while(p && ne[p][c] == q) ne[p][c] = nq , p = par[p];
            }
        }
    }
    void ini(){
        rt = last = L = 1;
        fill(ne[rt] , ne[rt] + M , 0);
        l[0] = -1;
    }
};
\end{lstlisting}
\subsection{SuffixTree}
\begin{lstlisting}[language=C++]
// init!! , go[0] is virtual , add 0 in the end of string
const int N = 101010 , C = 27 , inf = ~0U>>1;
int pos,S[N];
struct SuffixTree{
    struct Node{
        int l , r , du;
        Node *fail, *go[C], *fa;
        Node(int l=-1,int r=inf) : l(l),r(r){
            fail = fa = NULL; du = 0;
            memset(go,0,sizeof(go));
        }
        Node* link(Node*t){int c=S[t->l];du+=!go[c];go[c]=t;t->fa=this;return t;}
        int len(){return min(r,pos+1)-l;}
    }pool[N<<2],*pl,*rt,*p,*pre;
    int L,R;
    ll size; queue<Node*> leaves;
    void ini(){
        pos=-1;
        pl=pool;rt=p=new(pl++) Node(-1,-1);pre=NULL;
        L=R=0;
        size = 0; while(sz(leaves)) leaves.pop();
    }
    void jump(Node*u){
        if(pre) pre->fail = u;
        pre = u;
    }
    bool walk(Node*u){
        int len=u->len();
        if(R >= len) return L+=len,R-=len,p=u,true;
        return false;
    }
    void extend(int c){
        S[++pos] = c; pre = NULL;
        for(;;){
            int ch = S[L = R ? L : pos];
            if(p->go[ch]){
                Node*q = p->go[ch];
                if(walk(q)) continue;
                if(S[q->l + R] == c){ ++R; jump(p); break; }
                Node *s = new(pl++) Node(q->l,q->l+R);
                leaves.push(s->link(new(pl++) Node(pos)));
                q->l += R; p->link(s)->link(q);
                jump(s);
            }
            else leaves.push(p->link(new(pl++) Node(pos))) , jump(p);
            if(p == rt && !R) break;
            else if(p == rt) L = pos - --R;
            else p = p->fail ? p->fail : rt;
        }
        size += sz(leaves);
    }
    void eraseUp(Node*&u){
        size -= u->len();
        u->fa->go[S[u->l]] = NULL;
        --((u=u->fa)->du);
    }
    void erase(){
        Node*u = leaves.front(); leaves.pop();
        while(!u->du && u != p) eraseUp(u);
        if(u == p){
            if(!p->du && !R){
                L = pos - (R = p->len()) + 1;
                p = p->fa; eraseUp(u);
            }
            if(R && !p->go[S[L]]){
                Node *leaf = new(pl++) Node(L);
                leaves.push(p->link(leaf));
                size += leaf->len();
                if(p == rt && R) L = pos - --R + 1;
                else p = p->fail ? p->fail : rt;
            }
        }
    }
    int stop , ord[N<<1] , rk[N];
    void dfs(Node*u){
        ord[u - pool] = stop++;
        rep(i,0,C) if(u->go[i]) dfs(u->go[i]);
    }
    void getrk(){
        stop = 0;
        dfs(rt);
        for(int i=0;sz(leaves);++i)
            rk[i] = ord[leaves.front() - pool] , leaves.pop();
    }
};
\end{lstlisting}

\section{Tree}
\subsection{Centroid}
\begin{lstlisting}[language=C++]
// id starts from 1
namespace Centriod {
    const int N = 101010;
    int vis[N],sz[N];
    void dfssz(int c,int fa,int Sz,int &rt){
        sz[c] = 1;
        for(auto t : g[c]) if(!vis[t]&&t!=fa) dfssz(t,c,Sz,rt) , sz[c]+=sz[t];
        if(!rt && sz[c]*2>Sz) rt=c;
    }
    void dfs(int c){
        int rt=0;dfssz(c,0,0,rt);dfssz(c,0,sz[c],rt=0);
        // cal something
        vis[rt] = true;
        for(auto t : g[rt]) if(!vis[t]) dfs(t);
    }
};
\end{lstlisting}
\subsection{HeavyChain}
\begin{lstlisting}[language=C++]
// id starts with 1
struct HeavyChain{
    static const int N = 100005 , inf = ~0U>>1;
    int sz[N] , wson[N] , top[N] , dep[N] , id[N] , _ , par[N] , who[N];
    void dfs(int c,int fa,vi g[]){
        sz[c]=1;dep[c]=dep[par[c]=fa]+1;int &s=wson[c]=top[c]=0;
        for(auto t:g[c]) if(t!=fa)
            dfs(t,c,g),sz[c]+=sz[t],(sz[t]>=sz[s])&&(s=t);
    }
    void dfs2(int c,int fa,vi g[]){
        who[id[c]=++_]=c;int s=wson[c];
        if(!top[c]) top[c]=c;
        if(s) top[s]=top[c],dfs2(s,c,g);
        for(auto t:g[c]) if(t!=fa&&t!=s) dfs2(t,c,g);
    }
    void Query(int a,int b){// info in points
        int fa=top[a],fb=top[b];
        while(fa!=fb){
            if(dep[fa]<dep[fb]) swap(a,b) , swap(fa,fb);
            // Cal id[fa] .. id[a]
            a = par[fa];fa = top[a];
        }
        if(dep[a]<dep[b]) swap(a,b);
        // Cal id[b] .. id[a]
    }
    void Build(vi g[]){
        dfs(1,0,g);_=0;dfs2(1,0,g);
    }
}hc;
\end{lstlisting}
\subsection{LCARMQ}
\begin{lstlisting}[language=C++]
// N is 2 size of tree , id of nodes start from 0
struct LCARMQ{
    static const int N = 101010 << 1;
    int a[20][N] , lft[N] , dep[N] , lg[N] , L;
    int rmin(int x,int y){return dep[x] < dep[y] ? x : y;}
    void add(int x){ a[0][L++] = x;}
    void dfs(int c,int fa,const vi g[]){
        lft[c]=L;add(c);
        for(auto t : g[c]) if(t!=fa) dep[t]=dep[c]+1,dfs(t,c,g),add(c);
    }
    void Build(const vi g[]){
        L = 0;dfs(0,-1,g);
        rep(i,2,L) lg[i]=lg[i-1]+(i==(i&-i));
        rep(i,1,20){
            int lim = L+1-(1<<i);
            rep(j,0,lim) a[i][j] = rmin(a[i-1][j] , a[i-1][j+(1<<i>>1)]);
        }
    }
    int lca(int x,int y){
        x = lft[x] , y = lft[y];
        if(x > y) swap(x , y);
        int i = lg[y-x+1];
        return rmin(a[i][x] , a[i][y+1-(1<<i)]);
    }
};
\end{lstlisting}
\subsection{VTree}
\begin{lstlisting}[language=C++]
namespace Vtree{// some nodes remain
    const int N = 101010;
    int tp[N] , _;
    vi g[N];// nodes sorted in dfs order
    void solve(vi&v,LCARMQ&R){
        _ = 0;
        vi del;del.pb(tp[_++] = v[0]);
        rep(i,1,sz(v)){
            int lca = R.lca(tp[_-1] , v[i]);
            vi l;while(_ > 0 && R.dep[lca] < R.dep[tp[_-1]]) l.pb(tp[--_]);
            if(_ == 0 || lca != tp[_-1]) del.pb(tp[_++] = lca);
            l.pb(tp[_-1]);del.pb(tp[_++] = v[i]);
            rep(i,1,sz(l)) g[l[i]].pb(l[i-1]);
        }
        rep(i,0,_-1) g[tp[i]].pb(tp[i+1]);
        // dfs()
        for(auto t : del){
            // Cal()
            g[t].clear();
        }
    }
}
\end{lstlisting}

\section{ZProblems}
\subsection{Expression}
\begin{lstlisting}[language=C++]
char s[U<<2];
int len;
int a[U];
bool Q[6][6] = {
    0,0,0,0,0,0,
    1,1,0,0,0,0,
    1,1,1,0,0,0,
    1,1,1,0,0,0,
    1,1,1,1,0,0,
    0,0,0,0,0,0,
};

int id(char c){
    if(c == '~') return 0;
    if(c == '&') return 1;
    if(c == '|') return 2;
    if(c == '>') return 3;
    if(c == '=') return 4;
    // '('
    return 5;
}

int num[1010] , op[1010];
int ntop , otop;

void pop(){
    int c = op[otop--];
    int t1,t2;
    if(c == '~'){
        t1 = num[ntop];
        num[ntop] = imp[t1][1];
    }
    else if(c == '&'){
        t1 = num[ntop] , t2 = num[ntop-1];
        num[--ntop] = con[t2][t1];
    }
    else if(c == '|'){
        t1 = num[ntop] , t2 = num[ntop-1];
        num[--ntop] = dis[t2][t1];
    }
    else if(c == '>'){
        t1 = num[ntop] , t2 = num[ntop-1];
        num[--ntop] = imp[t2][t1];
    }
    else if(c == '='){
        int fr = ntop-1;
        while(otop != -1 && op[otop] == '=') fr-- , otop--;
        int k = -1,x,y,cal;
        for(int i=fr;i<ntop;++i){
            x = num[i] , y = num[i+1];
            cal = con[imp[x][y]][imp[y][x]];
            if(k == -1) k = cal;
            else k = con[k][cal];
        }
        num[ntop = fr] = k;
    }
}

bool ok(){
    ntop = otop = -1;
    rep(i,0,len){
        int c = s[i];
        if(c == ' ') continue;
        else if(isupper(c)) num[++ntop] = a[c-'A'];
        else if(isdigit(c)) num[++ntop] = c=='0';
        else if(c == ')'){
            while(op[otop] != '(') pop();
            otop--;
        }
        else{
            while(otop != -1 && Q[id(c)][id(op[otop])]) pop();
            op[++otop] = c;
        }
    }
    while(otop != -1) pop();
    return num[0] == 0;
}
\end{lstlisting}
\subsection{Geo}
\begin{lstlisting}[language=C++]
#include <iostream>
#include <vector>
#include <complex>
using namespace std;
#define rep(i,a,b) for(int i=a;i<b;++i)
#define sz(x) (int)(x).size()
namespace Geo{
    typedef double T;
    typedef complex<T> P;
    T dot(P a,P b){return (conj(a)*b).real();}
    T cross(P a,P b){return (conj(a)*b).imag();}
    T disq(P a,P b){return norm(a-b);}
    // able with integer above
    T dis(P a,P b){return abs(a-b);}
    T angle(P a,P b){return arg(b-a);}
    T angle(P a,P o,P b){return abs(remainder(arg(a-o)-arg(b-o),2.*M_PI));}
    P rot(P a,T theta){return a*polar(1.,theta);}
    P rot(P a,P o,T theta){return (a-o)*polar(1.,theta)+o;}
    P proj(P p,P v){return v*dot(p,v)/norm(v);}
    P proj(P p,P a,P b){return (b-a)*dot(p-a,b-a)/norm(b-a)+a;}
    P reflect(P p,P a,P b){return conj((p-a)/(b-a))*(b-a)+a;}
    P intersection(P a,P b,P p,P q){T c1=cross(p-a,b-a),c2=cross(q-a,b-a);
        return (c1*q-c2*p)/(c1-c2);}
}
using namespace Geo;
namespace NearestPoints{
    T Min;P tmp[100000];
    void Update(T disq) {Min = min(Min , disq);}
    T solve(int l,int r,P*p){
        if(l >= r) return 1e100;
        if(l + 1 == r){
            if(p[l].imag() > p[r].imag()) swap(p[l],p[r]);
            Update(disq(p[l],p[r]));
            return disq(p[l],p[r]);
        }
        int m=(l+r)>>1;
        T Xm = p[m].real() , lim = min(solve(l,m,p) , solve(m+1,r,p));
        int i=l,j=m+1,ii=m,jj=r,k=l;
        while(i<=ii&&j<=jj){
            while(i<=ii&&(j>jj||p[i].imag() <= p[j].imag())) tmp[k++] = p[i++];
            swap(i,j);swap(ii,jj);
        }
        vector<P> V;
        rep(i,l,r+1){
            p[i] = tmp[i];
            if(fabs(p[i].real() - Xm) <= lim) V.push_back(p[i]);
        }
        rep(i,0,sz(V)) rep(j,i+1,sz(V)){
            if(fabs(V[j].imag() - V[i].imag()) >= lim) break;
            T _dis = disq(V[i] , V[j]);
            Update(_dis);lim = min(lim,_dis);
        }
        return lim;
    }
    void ini(){Min = 1e100;}
}

int main(){
    return 0;
}
\end{lstlisting}
\subsection{PlugDP}
\begin{lstlisting}[language=C++]
/* 插头dp */
// 多回路
int g[12][12];
ll dp[12][12][1<<12];

#define b2(x) (1<<(x))
ll work(){
    scanf("%d%d",&n,&m);
    rep(i,1,n+1) rep(j,1,m+1) scanf("%d",&g[i][j]);
    rep(i,0,b2(m+1)) dp[0][m][i] = 0;
    dp[0][m][0] = 1;
    rep(i,1,n+1){
        rep(s,0,b2(m)) dp[i][0][s<<1] = dp[i-1][m][s];
        rep(j,1,m+1) rep(s,0,b2(m+1)){
            int p = 1<<j-1 , q = 1<<j;
            bool x = p&s , y = q&s;
            if(g[i][j]){
                dp[i][j][s] = dp[i][j-1][s^p^q];
                if(x^y) dp[i][j][s] += dp[i][j-1][s];
            }
            else{
                if(!(x||y)) dp[i][j][s] = dp[i][j-1][s];
                else dp[i][j][s] = 0;
            }
        }
    }
    return dp[n][m][0];
}

int main(){
    scanf("%d",&T);
    rep(i,1,T+1) printf("Case %d: There are %I64d ways to eat the trees.\n",i,work());
    return 0;
}
//	单回路
int n , m , end_n , end_m;
int g[15][15];
const int HASH = (1<<14) - 1 , NODE = 243 * 243 * 3;
int h[HASH + 1];
struct hashmap{
    int ne[NODE] , s[NODE] , e;
    ll sum[NODE];
    void ins(int p,ll d){
        int c = p&HASH;
        for(int k=h[c];~k;k=ne[k])
            if(s[k] == p){
                sum[k] += d;
                return;
            }
        s[e] = p , sum[e] = d;
        ne[e] = h[c];h[c] = e++;
    }
    void ini(){
        set(h,-1);
        e = 0;
    }
}hm[2];
int c;

ll work(){
    char s[20];
    rep(i,1,n+1){
        scanf("%s",s+1);
        rep(j,1,m+1) if(g[i][j] = s[j] == '.') end_n = i , end_m = j;
    }
    int t2[20];
    rep(i,0,20) t2[i] = i+i;
    c = 0;hm[c].ini();hm[c].ins(0,1);
    rep(i,1,n+1){
        rep(k,0,hm[c].e) hm[c].s[k] <<= 2;
        rep(j,1,m+1){
            c ^= 1;
            hm[c].ini();
            rep(k,0,hm[c^1].e){
                int s = hm[c^1].s[k];
                ll sum = hm[c^1].sum[k];
                if(s>>t2[m+1]) continue;
                int p = (s>>t2[j-1]) & 3 , q = (s>>t2[j]) & 3;
                if(g[i][j] == 0){
                    if(p == 0 && q == 0) hm[c].ins( s , sum );
                }
                else{
                    if(p == 0 && q == 0){
                        hm[c].ins( s ^ (1<<t2[j-1]) ^ (2<<t2[j]) , sum);
                    }
                    else if(p == 0 || q == 0){
                        hm[c].ins( s , sum );
                        hm[c].ins( s ^ ((p|q)<<t2[j-1]) ^ ((p|q)<<t2[j]) , sum );
                    }
                    else if(p == q){
                        int cnt = 1;
                        for(int k = p == 1 ?j+1:j-2;;p == 1 ?++k:--k){
                            int t = s>>t2[k]&3;
                            if(t == p) ++cnt;
                            if(t == (p^3)) --cnt;
                            if(cnt == 0){
                                hm[c].ins( s ^ (3<<t2[k]) ^ (p<<t2[j]) ^ (p<<t2[j-1]) , sum );
                                break;
                            }
                        }
                    }
                    else if(p == 1 && q == 2){
                        if(i == end_n && j == end_m)
                            hm[c].ins( s ^ (1<<t2[j-1]) ^ (2<<t2[j]) , sum);
                    }
                    else if(p == 2 && q == 1){
                        hm[c].ins( s ^ (2<<t2[j-1]) ^ (1<<t2[j]) , sum );
                    }
                }
            }
        }
    }
    rep(i,0,hm[c].e) if(hm[c].s[i] == 0) return hm[c].sum[i];
    return 0;
}

int main(){
    while(~scanf("%d%d",&n,&m)) printf("%I64d\n",work());
    return 0;
}
//	O->必选 X->障碍 *->可选
const int N = 20202<<1 , M = N << 1;
int T , n , m , end_n , end_m;
const int HASH = (1<<18)-1 , NODE = 243 * 243 * 3;
int h[HASH];
struct hashmap{
    int ne[NODE] , s[NODE] , e;
    ll sum[NODE];
    void ins(int val,ll d){
        int c = val & HASH;
        for(int k=h[c];~k;k=ne[k])
            if(s[k] == val){
                sum[k] += d;
                return;
            }
        s[e] = val , sum[e] = d;
        ne[e] = h[c];h[c] = e++;
    }
    void ini(){
        set(h,-1);
        e = 0;
    }
}hm[2];
char g[15][15];
int c;

ll work(){
    scanf("%d%d",&n,&m);
    rep(i,1,n+1){
        scanf("%s",g[i]+1);
        rep(j,1,m+1) if(g[i][j] == 'O')
            end_n = i , end_m = j;
    }
    ll ans = 0;
    bool lst = false;
    hm[c].ini();hm[c].ins(0,1);
    int t2[20];
    rep(i,0,20) t2[i] = i+i;
    rep(i,1,n+1){
        rep(k,0,hm[c].e) hm[c].s[k] <<= 2;
        rep(j,1,m+1){
            if(i == end_n && j == end_m) lst = true;
            c ^= 1;
            hm[c].ini();
            rep(k,0,hm[c^1].e){
                int s = hm[c^1].s[k];
                ll sum = hm[c^1].sum[k];
                if(s>>t2[m+1]) continue;
                int p = s>>t2[j-1]&3 , q = s>>t2[j]&3;
                if(g[i][j] == 'X'){
                    if(p == 0 && q == 0) hm[c].ins(s , sum);
                }
                else if(p == 0 && q == 0) hm[c].ins(s ^ 1<<t2[j-1] ^ 2<<t2[j] , sum);
                else if(p == 0 || q == 0) hm[c].ins(s , sum) , hm[c].ins(s ^ (p|q)<<t2[j-1] ^ (p|q)<<t2[j] , sum);
                else if(p == q){
                    int cnt = 1;
                    for(int k = p==1 ? j+1:j-2;;p==1?++k:--k){
                        int t = s>>t2[k]&3;
                        if(t == p) ++cnt;
                        if(t == (p^3)) --cnt;
                        if(cnt == 0){
                            hm[c].ins(s ^ 3<<t2[k] ^ p<<t2[j-1] ^ p<<t2[j] , sum);
                            break;
                        }
                    }
                }
                else if(p == 2 && q == 1) hm[c].ins(s ^ p<<t2[j-1] ^ q<<t2[j] , sum);
                else if(p == 1 && q == 2){
                    if(lst) if((s ^ p<<t2[j-1] ^ q<<t2[j]) == 0)
                        ans += sum;
                }
                if(g[i][j] == '*') if(p == 0 && q == 0) hm[c].ins(s , sum);
            }
        }
    }
    return ans;
}

int main(){
    scanf("%d",&T);
    rep(i,1,T+1) printf("Case %d: %I64d\n",i,work());
    return 0;
}
//	最小花费回路
const int N = 20202<<1 , M = N << 1;
const int HASH = 10007 , NODE = 243 * 243 * 3;
int h[HASH];
struct hashmap{
    int ne[NODE] , s[NODE] , e;
    int mi[NODE];
    void ins(int val,int t){
        int c = val % HASH;
        for(int k=h[c];~k;k=ne[k])
            if(s[k] == val){
                mi[k] = min(mi[k] , t);
                return;
            }
        s[e] = val , mi[e] = t;
        ne[e] = h[c];h[c] = e++;
    }
    void ini(){
        set(h,-1);
        e = 0;
    }
}hm[2];
int c;
char g[30][30];
int T , n , m;

int work(){
    scanf("%d%d",&n,&m);
    gets(g[0]);
    rep(i,1,2*n+2) gets(g[i] + 1);
    int w2[20];
    rep(i,0,20) w2[i] = i+i;
    hm[c].ini();hm[c].ins(0,0);
    rep(i,1,n+1){
        rep(k,0,hm[c].e) hm[c].s[k] <<= 2;
        rep(j,1,m+1){
            c ^= 1;
            hm[c].ini();
            rep(k,0,hm[c^1].e){
                int s = hm[c^1].s[k];
                int d = hm[c^1].mi[k];
                if(s>>w2[m+1]) continue;
                int p = s>>w2[j-1]&3 , q = s>>w2[j]&3;
                int up = g[i*2-1][j*2] - '0', left = g[i*2][j*2-1] - '0';
                if(p == 0 && q == 0) hm[c].ins(s ^ 1<<w2[j-1] ^ 2<<w2[j] , d);
                else if(p == 0 || q == 0) hm[c].ins(s , d + (p ? left : up)) ,
                    hm[c].ins(s ^ (p|q)<<w2[j-1] ^ (p|q)<<w2[j], d + (p ? left : up));
                else if(p == q){
                    int cnt = 1;
                    for(int k=p==1?j+1:j-2;;p==1?++k:--k){
                        int t = s>>w2[k]&3;
                        if(t == p) ++cnt;
                        if(t == (p^3)) --cnt;
                        if(cnt == 0){
                            hm[c].ins(s ^ 3<<w2[k] ^ p<<w2[j-1] ^ q<<w2[j] , d + left + up);
                            break;
                        }
                    }
                }
                else if(p == 2 && q == 1) hm[c].ins(s ^ p<<w2[j-1] ^ q<<w2[j] , d + left + up);
                else if(p == 1 && q == 2) if(i == n && j == m) hm[c].ins(s ^ p<<w2[j-1] ^ q<<w2[j] , d + left + up);
            }
        }
    }
    rep(i,0,hm[c].e) if(hm[c].s[i] == 0) return hm[c].mi[i];
    return 0;
}

int main(){
    scanf("%d",&T);
    rep(i,1,T+1) printf("%d\n",work());
    return 0;
}
//	左上到右下简单路径
const int N = 20202<<1 , M = N << 1;
const int HASH = 10007 , NODE = 243 * 243 * 3;
int h[HASH];
struct hashmap{
    int ne[NODE] , s[NODE] , e;
    int mi[NODE];
    void ins(int val,int t){
        int c = val % HASH;
        for(int k=h[c];~k;k=ne[k])
            if(s[k] == val){
                mi[k] = max(mi[k] , t);
                return;
            }
        s[e] = val , mi[e] = t;
        ne[e] = h[c];h[c] = e++;
    }
    void ini(){
        set(h,-1);
        e = 0;
    }
}hm[2];
int c;
int g[15][15];
bool can[15][15];
int T , n , m;

int work(){
    n += 2 , m += 2;
    set(g,0);set(can,true);
    rep(j,2,m) can[2][j] = false;
    rep(i,2,n) can[i][m-1] = false;
    rep(i,3,n+1) rep(j,1,m-1) scanf("%d",&g[i][j]);
    hm[c].ini(); hm[c].ins(0,0);
    int w2[20]; rep(i,0,20) w2[i] = i+i;
    rep(i,1,n+1){
        rep(k,0,hm[c].e) hm[c].s[k] <<= 2;
        rep(j,1,m+1){
            c ^= 1;
            hm[c].ini();
            rep(k,0,hm[c^1].e){
                int s = hm[c^1].s[k];
                int d = hm[c^1].mi[k];
                if(s>>w2[m+1]) continue;
                int p = s>>w2[j-1]&3 , q = s>>w2[j]&3;
                if(can[i][j] == 0){
                    if(p == 0 && q == 0)
                        hm[c].ins(s , d);
                }
                else if(p == 0 && q == 0){
                    if(i != 1 || j != 1) hm[c].ins(s , d);
                    hm[c].ins(s ^ 1<<w2[j-1] ^ 2<<w2[j] , d + g[i][j]);
                }
                else if(p == 0 || q == 0) hm[c].ins(s , d + g[i][j]) ,
                    hm[c].ins(s ^ (p|q)<<w2[j-1] ^ (p|q)<<w2[j] , d + g[i][j]);
                else if(p == q){
                    for(int cnt=1,k=p==1?j+1:j-2;;p==1?++k:--k){
                        int t = s>>w2[k]&3;
                        if(t == p) ++cnt;
                        if(t == (p^3)) --cnt;
                        if(cnt == 0){
                            hm[c].ins(s ^ 3<<w2[k] ^ p<<w2[j-1] ^ q<<w2[j] , d + g[i][j]);
                            break;
                        }
                    }
                }
                else if(p == 2 && q == 1) hm[c].ins(s ^ p<<w2[j-1] ^ q<<w2[j] , d + g[i][j]);
                else if(p == 1 && q == 2) if(i == n && j == m) hm[c].ins(s ^ p<<w2[j-1] ^ q<<w2[j] , d + g[i][j]);
            }
        }
    }
    rep(i,0,hm[c].e) if(hm[c].s[i] == 0) return hm[c].mi[i];
    return 0;
}

int main(){
    int _ = 0;
    while(~scanf("%d%d",&n,&m)) printf("Case %d: %d\n",++_,work());
    return 0;
}
//	0->空 1->障碍 把22,33不相交连起来最短
const int N = 20202<<1 , M = N << 1;
const int HASH = 10007 , NODE = 243 * 243 * 3;
int n , m , en , em;
int g[15][15];
int h[HASH];
struct hashmap{
    int mi[NODE] , ne[NODE] , e;
    int s[NODE];
    void ins(int x,int d){
        int c = x % HASH;
        for(int k=h[c];~k;k=ne[k])
            if(s[k] == x){
                mi[k] = min(mi[k] , d);
                return;
            }
        s[e] = x , mi[e] = d;
        ne[e] = h[c] ; h[c] = e++;
    }
    void ini(){
        set(h,-1);
        e = 0;
    }
}hm[2];
int c;

int work(){
    rep(i,1,n+1) rep(j,1,m+1){
        scanf("%d",&g[i][j]);
        if(g[i][j] >= 2) en = i , em = j;
    }
    hm[c].ini();hm[c].ins(0,0);
    int e3[20];
    rep(i,0,20) e3[i] = i+i+i;
    int ans = inf;
    rep(i,1,n+1){
        rep(k,0,hm[c].e){
            if(hm[c].s[k]>>e3[m]) hm[c].mi[k] = -1;
            else hm[c].s[k] <<= 3;
        }
        rep(j,1,m+1){
            c ^= 1;
            hm[c].ini();
            rep(k,0,hm[c^1].e){
                int s = hm[c^1].s[k];
                int d = hm[c^1].mi[k];
                if(s>>e3[m+1] || d < 0 || d >= ans) continue;
                if(s == 0 && (i >= en && j >= em || i > en))
                    ans = min(ans , d);
                int p = s>>e3[j-1]&7 , q = s>>e3[j]&7;
                if(g[i][j] == 1){
                    if(p == 0 && q == 0)
                        hm[c].ins(s , d);
                }
                else if(g[i][j] == 0){
                    if(p == 0 && q == 0) hm[c].ins(s , d) ,
                        hm[c].ins(s ^ 1<<e3[j-1] ^ 2<<e3[j] , d + 1);
                    else if(p == 0 || q == 0) hm[c].ins(s , d + 1) ,
                        hm[c].ins(s ^ (p|q)<<e3[j-1] ^ (p|q)<<e3[j] , d + 1);
                    else if(p == 2 && q == 1 || p == q && p >= 3)
                        hm[c].ins(s ^ p<<e3[j-1] ^ q<<e3[j] , d + 1);
                    else if(p == 1 && q == 2) continue;
                    else if(min(p,q) <= 2){
                        int a = min(p,q) , b = p^q^a;
                        for(int cnt=1,k=a==1?j+1:j-2;;a==1?++k:--k){
                            int t = s>>e3[k]&7;
                            if(t == a) ++cnt;
                            if(t == (3^a)) --cnt;
                            if(cnt == 0){
                                hm[c].ins(s ^ (3^a^b)<<e3[k] ^ p<<e3[j-1] ^ q<<e3[j] , d + 1);
                                break;
                            }
                        }
                    }
                }
                else if(g[i][j] >= 2){
                    int o = g[i][j] + 1;
                    if(p == 0 && q == 0) hm[c].ins(s ^ o<<e3[j-1], d + 1) ,
                        hm[c].ins(s ^ o<<e3[j] , d + 1);
                    else if(p == 0 || q == 0){
                        if(max(p,q) == o) hm[c].ins(s ^ p<<e3[j-1] ^ q<<e3[j] , d + 1);
                        else if(max(p,q) <= 2){
                            int a = max(p,q) , b = o;
                            for(int cnt=1,k=a==1?j+1:j-2;;a==1?++k:--k){
                                int t = s>>e3[k]&7;
                                if(t == a) ++cnt;
                                if(t == (3^a)) --cnt;
                                if(cnt == 0){
                                    hm[c].ins(s ^ (3^a^b)<<e3[k] ^ p<<e3[j-1] ^ q<<e3[j] , d + 1);
                                    break;
                                }
                            }
                        }
                    }
                }
            }
        }
    }
    rep(i,0,hm[c].e) if(hm[c].s[i] == 0){
        ans = min(hm[c].mi[i] , ans);
        break;
    }
    return ans == inf ? 0 : ans-2;
}


int main(){
    while(~scanf("%d%d",&n,&m)){
        if(!(n||m)) break;
        printf("%d\n",work());
    }
    return 0;
}
//	六边形多回路
const int N = 1<<17;
struct hashmap{
    ll sum[N];
    int q[N],e;
    void ins(int x,ll d){
        if(sum[x] == 0) q[e++] = x;
        sum[x] += d;
    }
    void ini(){
        rep(i,0,e) sum[q[i]] = 0;
        e = 0;
    }
}hm[2];int c;
int COL = 8;
int n , m;
int g[15][15];

void Trans(int y,bool f){
    c ^= 1;
    hm[c].ini();
    rep(k,0,hm[c^1].e){
        int s = hm[c^1].q[k] , ns=s;
        ll d = hm[c^1].sum[s];
        if(d == 0) continue;

        int cnt = 0;
        rep(j,0,3) if(s>>(y+y-j)&1) ns ^= 1<<(y+y-j) , ++cnt;
        if(f){
            if(cnt == 0) hm[c].ins(s , d);
        }
        else{
            if(cnt == 0) rep(i,0,2) rep(j,i+1,3) hm[c].ins(ns ^ 1<<(y+y-i) ^ 1<<(y+y-j) , d);
            if(cnt == 1) rep(i,0,3) hm[c].ins(ns ^ 1<<(y+y-i) , d);
            if(cnt == 2) hm[c].ins(ns , d);
        }
    }
}

ll work(){
    set(g,0);
    rep(i,0,m){
        char s[3];
        scanf("%s",s);
        g[s[0] - 'A' + 1][s[1] - 'A' + 1] = 1;
    }
    hm[c].ini();hm[c].ins(0,1);
    rep(i,1,n+1){
        rep(k,0,hm[c].e){
            int s = hm[c].q[k];
            if((s&1) || (s>>(COL+COL))) hm[c].sum[s] = 0;
        }
        for(int j=2;j<=COL;j+=2) Trans(j,g[i][j]);
        for(int j=1;j<=COL;j+=2) Trans(j,g[i][j]);
    }
    rep(i,0,hm[c].e) if(hm[c].q[i] == 0) return hm[c].sum[0];
    return 0;
}

int main(){
    while(~scanf("%d%d",&n,&m)) printf("%lld\n",work());
    return 0;
}
//	左上到左下,矩阵加速
const int N = 1<<16;
int v[N] , q[N] , _;
int w2[30];
int n , m;
struct hashmap{
    int v[N] , q[N] , e;
    void ins(int x,int d){
        if(v[x] == 0) q[e++] = x;
        v[x] += d;
    }
    void ini(){
        rep(i,0,e) v[q[i]] = 0;
        e = 0;
    }
}hm[2];int c;
struct MM{
    ll a[120][120];
    MM(ll x = 0){
        rep(i,0,_) rep(j,0,_) a[i][j] = i == j ? x : 0;
    }
    MM operator * (const MM& b){
        MM ret(0);
        rep(i,0,_) rep(j,0,_) rep(k,0,_) ret.a[i][j] += a[i][k] * b.a[k][j];
        rep(i,0,_) rep(j,0,_) ret.a[i][j] %= mod;
        return ret;
    }
    MM operator ^ (int r){
        MM ret(1) , sp = *this;
        while(r){
            if(r&1) ret = ret * sp;
            sp = sp * sp;
            r>>=1;
        }
        return ret;
    }
    void out(){
        rep(i,0,_) rep(j,0,_) printf("%I64d%c",a[i][j]," \n"[j==_-1]);
    }
}mm;

void Trans(int j,bool f){
    c ^= 1;
    hm[c].ini();
    rep(k,0,hm[c^1].e){
        int s = hm[c^1].q[k];
        int d = hm[c^1].v[s];
        int p = s>>w2[j-1]&3 , q = s>>w2[j]&3;
        if(p == 0 && q == 0) hm[c].ins(s ^ 1<<w2[j-1] ^ 2<<w2[j] , d);
        else if(p == 0 || q == 0) hm[c].ins(s , d) ,
            hm[c].ins(s ^ (p|q)<<w2[j-1] ^ (p|q)<<w2[j] , d);
        else if(p == 2 && q == 1) hm[c].ins(s ^ p<<w2[j-1] ^ q<<w2[j] , d);
        else if(p == 1 && q == 2 && f) hm[c].ins(s ^ p<<w2[j-1] ^ q<<w2[j] , d);
        else if(p == q){
            for(int cnt=1,k=p==1?j+1:j-2;;p==1?++k:--k){
                int t = s>>w2[k]&3;
                if(t == p) ++cnt;
                if(t == (p^3)) --cnt;
                if(cnt == 0){
                    hm[c].ins(s ^ 3<<w2[k] ^ p<<w2[j-1] ^ q<<w2[j], d);
                    break;
                }
            }
        }
    }
}
void dfs(int fr){
    hm[c].ini();hm[c].ins(fr,1);
    rep(j,1,m+1) Trans(j,0);
    rep(i,0,hm[c].e){
        int s = hm[c].q[i];
        int d = hm[c].v[s];
        if(s>>w2[m]) continue;
        s <<= 2;
        if(v[s] == -1) v[s] = _ , q[_++] = s;
        mm.a[v[fr]][v[s]] = d;
    }
}

void work(){
    if((n&1) && !(m&1)){
        puts("Impossible");
        return;
    }
    swap(n,m);
    rep(i,0,30) w2[i] = i+i;
    set(v,-1); _ = 0; v[1<<w2[1] ^ 2<<w2[m]] = 0;
    q[_++] = 1<<w2[1] ^ 2<<w2[m];
    set(mm.a,0);
    rep(i,0,_) dfs(q[i]);
    mm = mm ^ (n-1);
    hm[c].ini();
    rep(i,0,_) if(mm.a[0][i]) hm[c].ins(q[i] , mm.a[0][i]);
    rep(i,1,m) Trans(i,0);
    Trans(m,1);
    rep(i,0,hm[c].e) if(hm[c].q[i] == 0){
        printf("%d\n",hm[c].v[0] % mod);
        return;
    }
    printf("%d\n",0);
}

int main(){
    while(~scanf("%d%d",&n,&m)) work();
    return 0;
}
//	最大路径
const int N = 1<<18;
struct hashmap{
    int v[N] , q[N] , e;
    void ins(int x,int d){
        if(v[x] == -1) q[e++] = x;
        v[x] = max(v[x],d);
    }
    void ini(){
        rep(i,0,e) v[q[i]] = -1;
        e = 0;
    }
}hm[2];int c;
int T , n , m;
int g[10][10] , w2[30];
int ans;
inline void ins(int x,int y){
    hm[c].ins(x,y);
}

void Trans(int j,int dd){
    c ^= 1;
    hm[c].ini();
    rep(k,0,hm[c^1].e){
        int s = hm[c^1].q[k];
        int d = hm[c^1].v[s];
        int p = s>>w2[j-1]&3 , q = s>>w2[j]&3;
        if(dd == 0){
            if(p == 0 && q == 0) ins(s , d);
            continue;
        }
        if(p == 0 && q == 0) ins(s , d) ,
            ins(s ^ 1<<w2[j-1] ^ 2<<w2[j] , d + dd) ,
                ins(s ^ 3<<w2[j-1] , d + dd) ,
                ins(s ^ 3<<w2[j] , d + dd);
        else if(p == 0 || q == 0){
            ins(s , d + dd) ,
                ins(s ^ (p|q)<<w2[j-1] ^ (p|q)<<w2[j] , d + dd);
            int a = max(p,q);
            if(a == 3){
                if(!(s ^ p<<w2[j-1] ^ q<<w2[j])) ans = max(ans , d + dd);
            }
            else{
                for(int cnt=1,k=a==1?j+1:j-2;;a==1?++k:--k){
                    int t = s>>w2[k]&3;
                    if(t == a) ++cnt;
                    if(t == (a^3)) --cnt;
                    if(cnt == 0){
                        ins(s ^ a<<w2[k] ^ p<<w2[j-1] ^ q<<w2[j] , d + dd);
                        break;
                    }
                }
            }
        }
        else if(p == 2 && q == 1) ins(s ^ p<<w2[j-1] ^ q<<w2[j] , d + dd);
        else if(p == 1 && q == 2) continue;
        else if(min(p,q) <= 2){
            int a = min(p,q);
            for(int cnt=1,k=a==1?j+1:j-2;;a==1?++k:--k){
                int t = s>>w2[k]&3;
                if(t == a) ++cnt;
                if(t == (a^3)) --cnt;
                if(cnt == 0){
                    ins(s ^ (3^p^q)<<w2[k] ^ p<<w2[j-1] ^ q<<w2[j] , d + dd);
                    break;
                }
            }
        }
        else{
            if(!(s ^ p<<w2[j-1] ^ q<<w2[j])) ans = max(ans , d + dd);
        }
    }
}

int work(){
    scanf("%d%d",&n,&m);
    ans = 0;
    rep(i,1,n+1) rep(j,1,m+1) scanf("%d",&g[i][j]) , ans = max(ans , g[i][j]);
    rep(i,0,30) w2[i] = i+i;
    hm[c].ini();hm[c].ins(0,0);
    rep(i,1,n+1){
        c ^= 1;
        hm[c].ini();
        rep(k,0,hm[c^1].e){
            int s = hm[c^1].q[k];
            if(!(s>>w2[m])) ins(s<<2,hm[c^1].v[s]);
        }
        rep(j,1,m+1) Trans(j,g[i][j]);
    }
    return ans;
}

int main(){
    scanf("%d",&T);
    set(hm[0].v,-1);set(hm[1].v,-1);
    rep(i,0,T) printf("%d\n",work());
    return 0;
}
//	K回路无环套环
const int N = 50505 , M = N<<1;
inline void pp(int &x,int d){
    x += d;if(x >= mod) x -= mod;
}
int T,n,m,K;
char g[15][15];
const int HASH = 10007 , NODE = 15511;
struct hashmap{
    int h[HASH] , ne[NODE] , s[NODE] , sum[NODE] , e;
    void ins(int x,int d){
        int c = x % HASH;
        for(int k=h[c];~k;k=ne[k])
            if(s[k] == x){
                pp(sum[k],d);
                return;
            }
        s[e] = x , sum[e] = d;
        ne[e] = h[c] , h[c] = e++;
    }
    void ini(){
        set(h,-1);
        e = 0;
    }
};
hashmap hm[2][37];int c;
void wk(){
    scanf("%d%d%d",&n,&m,&K);
    rep(i,1,n+1) scanf("%s",g[i]+1);
    int ex(-1) , ey(-1);
    rep(i,1,n+1) rep(j,1,m+1){
        if(g[i][j] == '.') ex = i , ey = j;
    }
    if(ex == -1 || K == 0){
        if(ex == -1 && K == 0) printf("1\n");
        else printf("0\n");
        return ;
    }
    int w2[20]; rep(i,0,20) w2[i] = i+i;
    rep(i,0,K) hm[c][i].ini();
    hm[c][0].ins(0,1);
    int ans = 0;
    rep(i,1,n+1){
        rep(j,0,K) rep(k,0,hm[c][j].e){
            if(hm[c][j].s[k] >> w2[m]) hm[c][j].sum[k] = 0;
            else hm[c][j].s[k] <<= 2;
        }
        if(i > ex) break;
        rep(j,1,m+1){
            if(i == ex && j > ey) break;
            c ^= 1;
            rep(o,0,K) hm[c][o].ini();
            rep(o,0,K){
                rep(k,0,hm[c^1][o].e){
                    int s = hm[c^1][o].s[k];
                    int d = hm[c^1][o].sum[k];
                    if(d == 0) continue;
                    //					cout << o << "  " << i << "  " << j << "  " << bitset<10>(s) << "  " << d << endl;
                    int p = s>>w2[j-1]&3 , q = s>>w2[j]&3 , ns = s^p<<w2[j-1]^q<<w2[j];
                    if(g[i][j] == '*'){
                        if(p == 0 && q == 0) hm[c][o].ins(s , d);
                    }
                    else if(p == 0 && q == 0) hm[c][o].ins(s ^ 1<<w2[j-1] ^ 2<<w2[j] , d);
                    else if(p == 0 || q == 0) hm[c][o].ins(s ^ (p|q)<<w2[j-1] ^ (p|q)<<w2[j] , d) ,
                        hm[c][o].ins(s , d);
                    else if(p == 2 && q == 1) hm[c][o].ins(ns, d);
                    else if(p == 1 && q == 2){
                        int cnt = 0;
                        rep(k,0,j-1) if(s>>w2[k]&3) ++cnt;
                        if(cnt&1) continue;
                        if(o == K-1){
                            if(ns == 0 && i == ex && j == ey) pp(ans,d);
                        }
                        else hm[c][o+1].ins(ns , d);
                    }
                    else if(p == q){
                        for(int cnt=1,k=p==1?j+1:j-2;;p==1?++k:--k){
                            int t = s>>w2[k]&3;
                            if(t == p) ++cnt;
                            if(t == (p^3)) --cnt;
                            if(cnt == 0){
                                hm[c][o].ins(ns ^ 3<<w2[k], d);
                                break;
                            }
                        }
                    }
                }
            }
        }
    }
    printf("%d\n",ans);
}

int main(){
    scanf("%d",&T);
    rep(i,0,T) wk();
    return 0;
}
\end{lstlisting}
\subsection{Poj4050}
\begin{lstlisting}[language=C++]
#include <iostream>
#include <cstdio>
#include <map>
using namespace std;
typedef long long ll;
const int Q = 1e4 + 7 , P = Q * Q , N = 100001;
int n , m;
int pri[N] , phi[N] , vis[N];
void Phi(){
    phi[1] = 1;
    for(int i=2,p=0;i<N;++i){
        if(!vis[i]) pri[p++] = i , phi[i] = i-1;
        for(int j=0,o;j<p&&(o=pri[j]*i)<N;++j){
            vis[o] = true;
            phi[o] = phi[i] * (pri[j] - 1);
            if(i % pri[j] == 0){
                phi[o] = phi[i] * pri[j];
                break;
            }
        }
    }
}
void pp(int&a,int b){
    a += b;
    if(a >= P) a -= P;
}
int Pow(int a,int b){ll r=1;for(;b;b>>=1,a=(ll)a*a%P)if(b&1)r=(ll)r*a%P;return r;}
typedef map<int,int> Poly;
ll a[100] , b[100] , _;int cnt[4];char s[200];
void out(Poly a){
    cout << "~~~~~~~~~~~~~" << endl;
    for(Poly::iterator i = a.begin();i != a.end();++i)
        cout << i->first << " " << i->second <<endl;
}
Poly operator * (Poly a,Poly b){
    Poly r;
    for(Poly::iterator i = a.begin();i != a.end();++i)
        for(Poly::iterator j = b.begin();j != b.end();++j)
            pp(r[(i->first + j->first) % P] , (ll)i->second * j->second % P);
    return r;
}
ll solve(int x){
    if(n / x % 2 == 0 && cnt[0]) return 0;
    if(cnt[0] + cnt[2] > x) return 0;
    int ret = 0;
    for(int i=0;i<_;++i) pp(ret , (ll)Pow(a[i] , x) * b[i] % P);
    return ret;
}
int main(){
    Phi();
    while(~scanf("%d%d",&m,&n)&&(n||m)){
        scanf("%s",s);fill(cnt , cnt + 4 , 0);int Magic = 1;
        for(int i=0;i<m;++i) cnt[s[i] - 'A']++ , (s[i] - 'A' <= 1 && (Magic<<=1));
        Poly Po , p[4];Po[0] = 1;p[0][1] = p[1][P-1] = p[1][1] = 1;p[0][P-1] = P-1;
        p[2][1] = p[3][1] = 1;p[2][0] = P-1;
        for(int i=0;i<m;++i) Po = Po * p[s[i] - 'A'];
        _ = 0;
        for(Poly::iterator i = Po.begin();i != Po.end();++i) if(i->second) a[_] = i->first , b[_++] = i->second;
        int ans = 0;
        for(int i=1;i*i<=n;++i) if(n % i == 0){
            pp(ans , (ll)solve(i) * phi[n/i] % P);
            if(i * i != n) pp(ans , (ll)solve(n/i) * phi[i] % P);
        }
        ans = (ll)ans * Pow(Magic , P-2) % P;
        if(n % Q == 0) ans = (ll)ans * Pow(n / Q , P - 2) % P / Q;
        else ans = (ll)ans * Pow(n , P - 2) % Q;
        printf("%d\n",(int)ans);
    }
    return 0;
}
\end{lstlisting}
\subsection{hull}
\begin{lstlisting}[language=C++]
#include <algorithm>
#include <iostream>
#include <cstring>
#include <vector>
#include <cstdio>
#include <string>
#include <cmath>
#include <queue>
#include <set>
#include <map>
#include <complex>
using namespace std;
typedef long long ll;
typedef double db;
typedef pair<int,int> pii;
typedef vector<int> vi;
#define de(x) cout << #x << "=" << x << endl
#define rep(i,a,b) for(int i=a;i<(b);++i)
#define all(x) (x).begin(),(x).end()
#define sz(x) (int)(x).size()
#define mp make_pair
#define pb push_back
#define fi first
#define se second
#define setIO(x) freopen(x".in","r",stdin);freopen(x".out","w",stdout)
const int N = 50001 * 3;
typedef complex<ll> P;
typedef map<ll,P> hull;
#define X real()
#define Y imag()
P a[N] , b[N] , O;int _;
int A(const P&a){ return (a.X >= 0) ^ ((a.Y < 0) + 1);}
ll cross(const P&a,const P&b){return (conj(a)*b).Y;}
bool Cmp(const P&a,const P&b){
    int x = A(a) , y = A(b);
    if(x != y) return x < y;
    return cross(a,b) > 0;
}
hull h1,h2;
bool in(hull&h,ll x,ll y){
    if(!sz(h)) return false;
    if(x < h.begin()->se.X || x > h.rbegin()->se.X) return false;
    auto l = h.lower_bound(x);
    if(x == l->se.X) return y <= l->se.Y;
    auto r = l--;
    return cross(r->se - l->se , P(x,y) - l->se) <= 0;
}

void ins(hull&h,ll x,ll y){
    if(in(h , x , y)) return;
    P p(x,y);h[x] = p;
    auto LL = h.find(x) , RR = LL , L = LL , R = L;
    if(LL != h.begin()) --LL;
    if(*RR != *h.rbegin()) ++RR;
    if(L != h.begin()) while((L = LL) != h.begin()){
        --(LL = L);
        if(cross(p - LL->se , L->se - LL->se) <= 0) h.erase(L);
        else break;
    }
    if(*R != *h.rbegin()) while(*(R = RR) != *h.rbegin()){
        ++(RR = R);
        if(cross(p - RR->se , R->se - RR->se) >= 0) h.erase(R);
        else break;
    }
}
int main(){
    _ = 0;O = P(0,0);
    rep(re,0,3){
        int n;scanf("%d",&n);
        int id = 0;
        rep(i,0,n){
            int x,y;scanf("%d%d",&x,&y);
            a[i] = P(x,y);
            if(a[i].Y < a[id].Y || (a[i].Y == a[id].Y && a[i].X < a[id].X)) id = i;
        }
        O += a[id];a[n] = a[0];
        int __ = _;
        rep(i,id,n) b[__++] = a[i+1] - a[i];
        rep(i,0,id) b[__++] = a[i+1] - a[i];
        inplace_merge(b,b+_,b+__,Cmp);
        _ = __;
    }
    a[0] = O;rep(i,0,_) a[i+1] = a[i] + b[i];
    rep(i,0,_) ins(h1 , a[i].X , a[i].Y) , ins(h2 , a[i].X , -a[i].Y);
    int m;scanf("%d",&m);
    rep(i,0,m){
        int x , y;scanf("%d%d",&x,&y);x*=3;y*=3;
        puts(in(h1,x,y) && in(h2,x,-y) ? "YES" : "NO");
    }
    return 0;
}
\end{lstlisting}
\subsection{三维几何}
\begin{lstlisting}[language=C++]
/*	@5388 三维几何
	T l r x y z, translate points;
	S l r x y z k, scale points;
	R r x y z x' y' z' a, rotate points;
	C i, calculate the current coordinates of the point;
	L l r, calculate the current length of the polygonal chain.
*/


const int N = (222222 + 10);
inline int sgn(db x){
	return fabs(x) < eps ? 0 : (x > 0 ? 1 : -1);
}
struct MM{
	db a[4][4];
	MM(db x=1) {rep(i,0,4) rep(j,0,4) a[i][j] = i == j ? x : 0;}
	void in() {*this = MM(); rep(i,0,3) scanf("%lf",&a[0][i]);a[0][3] = 1;}
	void out() {rep(i,0,3) printf("%.5lf%c",a[0][i]," \n"[i==2]);}
	db dis() {db ret(0); rep(i,0,3) ret += a[0][i]*a[0][i]; return sqrt(ret);}
	MM operator -= (const MM&b) {rep(i,0,4) rep(j,0,4) a[i][j] -= b.a[i][j];return *this;}
	MM operator *= (const MM&b) {MM tmp(0);rep(i,0,4) rep(j,0,4) rep(k,0,4) tmp.a[i][j] += a[i][k] * b.a[k][j]; return *this = tmp;}
	bool operator == (const MM&b) {rep(i,0,4) rep(j,0,4) if(sgn(a[i][j] - b.a[i][j])) return false; return true;}
	MM(db x,db y,db z){ // Translation
		*this = MM(1); a[3][0] = x , a[3][1] = y , a[3][2] = z;
	}
	MM(db x ,db y,db z,db k){ // Scaling
		*this = MM(k); a[3][0] = (1-k)*x , a[3][1] = (1-k)*y , a[3][2] = (1-k)*z , a[3][3] = 1;
	}
	MM(db x ,db y,db z,db xx,db yy,db zz,double aa){ // Rotation
		MM t1 = MM(-x,-y,-z) , t2 = MM(x,y,z);
		db d = sqrt(xx*xx+yy*yy+zz*zz); x = xx/d , y = yy/d , z = zz/d;
		db COS = cos(aa) , SIN = sin(aa);
		*this = MM(1);
		a[0][0] = x*x*(1-COS) + COS , a[0][1] = x*y*(1-COS) + z*SIN	 , a[0][2] = x*z*(1-COS) - y*SIN;
		a[1][0] = x*y*(1-COS) - z*SIN,a[1][1] = y*y*(1-COS) + COS   , a[1][2] = y*z*(1-COS) + x*SIN;
		a[2][0] = x*z*(1-COS) + y*SIN,a[2][1] = y*z*(1-COS)	- x*SIN , a[2][2] = z*z*(1-COS) + COS;
		t1 *= *this , t1 *= t2;
		*this = t1;
	}
	void oo(){
		rep(i,0,4) rep(j,0,4) printf("%.5lf%c",a[i][j]," \n"[j==3]);
	}
}I,a[N<<2];
#define pos int x,int l,int r
#define lson x<<1
#define rson x<<1|1
#define left lson,l,m
#define right rson,m+1,r
#define getm int m=(l+r)>>1
double c[N<<2] , ans[N<<2];
void Build(pos){
	a[x] = I; c[x] = 1;
	if(l == r){
		a[x].in();
		return;
	}getm;
	Build(left);Build(right);
}
inline void downA(int x){
	if(!(a[x] == I)) a[lson] *= a[x] ,
		a[rson] *= a[x] , a[x] = I;
}
MM Query(pos,int p){
	if(l == r) return a[x];
	getm;downA(x);
	if(p <= m) return Query(left,p); else return Query(right,p);
}
void Modify(pos,int L,int R,MM b){
	if(L <= l && r <= R){
		a[x] *= b;
		return;
	}getm;downA(x);
	if(L <= m) Modify(left,L,R,b);
	if(R > m) Modify(right,L,R,b);
}
inline void downC(int x){
	if(c[x] != 1) ans[lson] *= c[x] , c[lson] *= c[x] ,
		ans[rson] *= c[x] , c[rson] *= c[x] , c[x] = 1;
}
void Change(pos,int p,double v){
	if(l == r){
		ans[x] = v;
		return;
	}getm;downC(x);
	if(p <= m) Change(left,p,v); else Change(right,p,v);
	ans[x] = ans[lson] + ans[rson];
}
void mult(pos,int L,int R,double k){
	if(L <= l && r <= R){
		c[x] *= k , ans[x] *= k;
		return;
	}getm;downC(x);
	if(L <= m) mult(left,L,R,k);
	if(R > m) mult(right,L,R,k);
	ans[x] = ans[lson] + ans[rson];
}
double sum(pos,int L,int R){
	if(L <= l && r <= R) return ans[x];
	getm;downC(x);
	double ret = 0;
	if(L <= m) ret += sum(left,L,R);
	if(R > m) ret += sum(right,L,R);
	return ret;
}

int n , m;
int main(){
	int re = 0;
	while(~scanf("%d%d",&n,&m)){
		if(n == 0 && m == 0) return 0;
		printf("Case #%d:\n",++re);
		Build(1,1,n);
		rep(i,1,n) Change(1,1,n,i,(Query(1,1,n,i)-=Query(1,1,n,i+1)).dis());
		char s[100];
		double x , y , z , xx , yy , zz , k , theta;
		int l , r;
		rep(i,0,m){
			scanf("%s",s);
			scanf("%d",&l);
			if(s[0] == 'C')
				Query(1,1,n,l).out();
			else{
				scanf("%d",&r);
				if(s[0] == 'L'){
					printf("%.5lf\n",l+1<=r?sum(1,1,n,l,r-1):0);
				}
				else{
					scanf("%lf%lf%lf",&x,&y,&z);
					if(s[0] == 'T') Modify(1,1,n,l,r,MM(x,y,z));
					else if(s[0] == 'S'){
						scanf("%lf",&k);
						Modify(1,1,n,l,r,MM(x,y,z,k));
						if(l < r) mult(1,1,n,l,r-1,k);
					}
					else if(s[0] == 'R'){
						scanf("%lf%lf%lf%lf",&xx,&yy,&zz,&theta);
						Modify(1,1,n,l,r,MM(x,y,z,xx,yy,zz,theta));
					}
					if(l > 1) Change(1,1,n,l-1,(Query(1,1,n,l-1)-=Query(1,1,n,l)).dis());
					if(r < n) Change(1,1,n,r,(Query(1,1,n,r)-=Query(1,1,n,r+1)).dis());
				}
			}
		}
	}
	return 0;
}
\end{lstlisting}
\subsection{二分图计数}
\begin{lstlisting}[language=C++]
/*	@Gym-100524B
    带标号2染色二分图种类数 2-->x
    */
const int N = 101 , M = N<<1;
int n;
ll a[N] , b[N] , c[N] , C[N][N] , e3[N*N];
inline void pp(ll &x,ll d){
    x += d; if(x >= mod) x -= mod;
}
int main(){
    rep(i,0,N) rep(j,0,i+1)
        C[i][j] = j ? (C[i-1][j-1] + C[i-1][j]) % mod : 1;
    e3[0] = 1;
    rep(i,1,N*N) e3[i] = e3[i-1] * 3 % mod;
    a[0] = 1;
    rep(i,1,N) rep(j,1,i+1)
        pp(a[i] , e3[j*(i-j)] * C[i-1][j-1] % mod);
    rep(i,0,N){
        b[i] = a[i];
        rep(j,1,i) pp(b[i] , mod - C[i-1][j-1] * b[j] % mod
                * a[i-j] % mod * 2 % mod);
    }
    c[0] = 1;
    rep(i,1,N) rep(j,1,i+1)
        pp(c[i] , C[i-1][j-1] * b[j] % mod * c[i-j] % mod);
    while(~scanf("%d",&n),n) printf("%I64d\n",c[n]);
    return 0;
}
\end{lstlisting}
\subsection{动态区间第K大}
\begin{lstlisting}[language=C++]
/*	@zoj-2112
    动态区间第K大
    */
struct Index:vector<int>{
    void Pre(){
        sort(begin(),end());
        erase(unique(begin(),end()),end());
    }
    int rk(int x){
        return lower_bound(begin(),end(),x) - begin() + 1;
    }
}V;
struct Request{
    char op;
    int a , b , kth;
}R[M];
int T , n , m , a[N];
struct node{
    int l,r,w;
}po[2500005];
#define l(x) po[x].l
#define r(x) po[x].r
#define w(x) po[x].w
#define pos int l,int r,int &c
#define lson l,m,l(c)
#define rson m+1,r,r(c)
#define getm int m=l+r>>1
#define lb(x) (x)&(-x)
struct Ztree{
    int L , len , rt[N<<1];
    void ini(){
        L = 0;
        Set(rt,0);
    }
    void ins(int x,pos){
        po[++L] = po[c]; po[c = L].w++;
        if(l == r)return;
        getm;if(x <= m) ins(x,lson);
        else ins(x,rson);
    }
    void add(int x,int d,pos){
        if(c == 0) po[++L] = po[c] , c = L;
        w(c) += d;
        if(l == r) return;
        getm;if(x <= m) add(x,d,lson);
        else add(x,d,rson);
    }
    void Add(int x,int d,int i){
        for(;i<=n;i+=lb(i))
            add(x,d,1,len,rt[i]);
    }vi pl , ml;
    int query_kth(int l,int r,int kth){
        if(l == r) return l;getm;
        int cnt = 0;
        rep(i,0,sz(pl)) cnt += w(l(pl[i]));
        rep(i,0,sz(ml)) cnt -= w(l(ml[i]));
        rep(i,0,sz(pl)) pl[i] = cnt >= kth ? l(pl[i]) : r(pl[i]);
        rep(i,0,sz(ml)) ml[i] = cnt >= kth ? l(ml[i]) : r(ml[i]);
        if(cnt >= kth) return query_kth(l,m,kth);
        else return query_kth(m+1,r,kth - cnt);
    }
    int Query_kth(int l,int r,int kth){
        pl.clear();ml.clear();
        ml.push_back(rt[l == 1 ? 0 : n+l-1]);
        pl.push_back(rt[n + r]);
        for(int i=l-1;i>=1;i-=lb(i)) ml.push_back(rt[i]);
        for(int i=r;i>=1;i-=lb(i)) pl.push_back(rt[i]);
        return query_kth(1,len,kth);
    }
}Z;
void work(){
    scanf("%d%d",&n,&m);
    V.clear();
    rep(i,1,n+1) scanf("%d",a + i) , V.pb(a[i]);
    char op[10];
    rep(i,0,m){
        scanf("%s%d%d",op,&R[i].a,&R[i].b);
        R[i].op = op[0];
        if(op[0] == 'Q') scanf("%d",&R[i].kth);
        else V.pb(R[i].b);
    }
    V.Pre();Z.len = sz(V);Z.ini();
    rep(i,1,n+1) Z.ins(V.rk(a[i]),1,Z.len,Z.rt[i+n] = Z.rt[i+n-1]);
    rep(i,0,m){
        if(R[i].op == 'Q') printf("%d\n",V[Z.Query_kth(R[i].a,R[i].b,R[i].kth) - 1]);
        else{
            Z.Add(V.rk(a[R[i].a]) ,			-1 , R[i].a);
            Z.Add(V.rk(a[R[i].a] = R[i].b) , 1 , R[i].a);
        }
    }
}
\end{lstlisting}
\subsection{双并查集}
\begin{lstlisting}[language=C++]
#include <iostream>
#include <cstring>
#include <cstdlib>
#include <cstdio>
#include <cmath>
#include <ctime>
#include <queue>
#include <set>
#include <map>
#include <string>
#include <bitset>
#include <vector>
#include <complex>
#include <algorithm>
using namespace std;
typedef long long ll;
typedef double db;
typedef pair<int,int> pii;
typedef vector<int> vi;
#define de(x) cout << #x << "=" << x << endl
#define rep(i,a,b) for(int i=a;i<(b);++i)
#define all(x) (x).begin(),(x).end()
#define sz(x) (int)(x).size()
#define mp make_pair
#define pb push_back
#define fi first
#define se second
const int N = 300003;
int n , m , p , par[N] , sz[N] , wson[N] , top[N] , dep[N] , edgecnt[N];
int A[N] , B[N];
struct Event{
    int w,u,v,id;
}event[N];
vector<pii> fobid[N] , block;
int FA(int x){return A[x]==x?x:A[x]=FA(A[x]);}
int FB(int x){return B[x]==x?(!x||FA(x)!=FA(par[x])?x:B[x]=FB(par[x])):B[x]=FB(B[x]);}
int Lca(int x,int y){
    while(top[x]!=top[y]) dep[top[x]]>dep[top[y]]?x=par[top[x]]:y=par[top[y]];
    return dep[x]<dep[y]?x:y;
}

int main(){
    scanf("%d%d%d",&n,&m,&p);
    rep(i,2,n+1) scanf("%d",par + i),dep[i]=dep[par[i]]+1;
    rep(i,1,n+1) A[i]=B[i]=i;
    for(int i=n;i>=2;--i) sz[par[i]]+=++sz[i],(sz[i]>sz[wson[par[i]]])&&(wson[par[i]]=i);
    rep(i,1,n+1) top[i]=wson[par[i]]==i?top[par[i]]:i;
    rep(i,0,m) scanf("%d%d%d",&event[i].u,&event[i].v,&event[i].w),event[i].id=i;
    int t,a,b;rep(i,0,p) scanf("%d%d%d",&t,&a,&b) , fobid[--t].pb(mp(a,b));
    sort(event,event+m,[&](const Event&a,const Event&b){return a.w<b.w;});
    ll ans = 0;
    rep(i,0,m){
        int u=event[i].u,v=event[i].v,w=event[i].w,id=event[i].id;
        for(auto&e:fobid[id]){
            e.fi = FA(e.fi) , e.se = FA(e.se);
            if(e.fi < e.se) swap(e.fi,e.se);
        }
        sort(all(fobid[id]));
        vector<pii> block , preblock , nxtblock;
        int lca=Lca(u,v);
        for(;FB(u)!=FB(lca);u=par[FB(u)])
            block.pb(mp(FA(u) , dep[u]-dep[FB(u)]+1));
        for(;FB(v)!=FB(lca);v=par[FB(v)])
            block.pb(mp(FA(v) , dep[v]-dep[FB(v)]+1));
        block.pb(mp(FA(lca) , dep[u]+dep[v]-2*dep[lca]+1));
        sort(all(block));
        for(int i=0,j=0,k=0;i<sz(block);i=j){
            pii info = mp(block[j].fi , 0);
            for(j=i;j<sz(block) && block[j].fi==block[i].fi;++j)
                info.se += block[j].se;
            for(;k<sz(fobid[id])&&fobid[id][k].fi == info.fi;++k)
                edgecnt[FA(fobid[id][k].se)]++;
            int Sum = 0;
            for(auto e : preblock){
                if(edgecnt[e.fi] < 1ll * e.se * info.se)
                    A[e.fi] = info.fi , Sum += e.se , ans += w;
                else nxtblock.pb(e);
                edgecnt[e.fi] = 0;
            }
            edgecnt[info.fi] = 0;
            nxtblock.pb(mp(info.fi , info.se + Sum));
            swap(preblock , nxtblock);nxtblock.clear();
        }
    }
    printf("%lld\n",ans);
    return 0;
}
\end{lstlisting}
\subsection{后缀数组}
\begin{lstlisting}[language=C++]
/* 后缀数组 */
// repeat string

int T , n;
int a[N];
int ans;

void go(int d){
	for(int l=0;l+d<n;l+=d){
		int gt = lcp(l,l+d);
		int tmp = gt / d;
		int jp = l - (d - gt%d);
		if(jp >= 0) tmp = max(tmp , lcp(jp,jp+d) / d);
		ans = max(tmp + 1 , ans);
	}
}

int main(){
	scanf("%d",&T);
	rep(re,0,T){
		scanf("%d",&n);
		rep(i,0,n){
			char c[2];
			scanf("%s",c);
			a[i] = c[0] - 'a' + 1;
		}a[n] = 0;
		da(a,n+1,100);cal_h(a,n);rmq(n);
		ans = 1;
		rep(i,1,n+1) go(i);
		printf("%d\n",ans);
	}
	return 0;
}
//	字典序最小,repeat string
void go(int d){
	int tmp = 0;
	for(int l=0;l+d<len;l+=d){
		int gt = lcp(l,l+d);
		tmp = max(tmp , gt / d);
		int jp = l - (d - gt%d);
		if(jp >= 0) tmp = max(tmp , lcp(jp,jp+d) / d);
	}
	if(tmp > ans) _q = 0 , ans = tmp;
	if(tmp >= ans) q[_q++] = d;
}

int re = 0;
void out(){
	for(int i=1;i<=len;++i){
		int id = sa[i];
		for(int j=0;j<_q;++j){
			if(id + q[j] >= len) continue;
			if(lcp(id,id+q[j]) >= ans * q[j]){
				s[id + (ans + 1) * q[j]] = 0;
				printf("Case %d: ",++re);
				puts(s + id);
				return;
			}
		}
	}
}

int main(){
	while(~scanf("%s",s)){
		len = strlen(s);
		if(len == 1 && s[0] == '#') return 0;
		rep(i,0,len) a[i] = s[i] - 'a' + 100;
		a[len] = 0;
		da(a,len+1,200); cal_h(a,len); rmp(len);
		ans = 0; rep(i,1,len+1) go(i);
		out();
	}
	return 0;
}
//	长度不小于K的公共串对
char s[N];
int a[N] , n , len , K;
int who[N];
ll ans;
pii st[N];
int p;
void cal(int p1,int p2){
	p = -1;
	ll tot = 0;
	rep(i,1,n+1){
		if(h[i] < K) p = -1 , tot = 0;
		else{
			int cnt = 0;
			if(who[sa[i-1]] == p1) cnt++ , tot += h[i] - K + 1;
			while(~p && h[i] <= st[p].fi){
				tot -= 1ll * st[p].se * (st[p].fi - h[i]);
				cnt += st[p].se;
				--p;
			}
			st[++p] = mp(h[i],cnt);
			if(who[sa[i]] == p2) ans += tot;
		}
	}
}


int main(){
	while(~scanf("%d",&K)){
		if(K == 0) return 0;
		scanf("%s",s);
		n = len = strlen(s);
		rep(i,0,len) a[i] = s[i] , who[i] = 1;
		a[n] = 1; who[n++] = 0;
		scanf("%s",s);
		rep(i,0,len) a[n+i] = s[i] , who[n+i] = 2;
		a[n += len] = 0; who[n] = 0;
		da(a,n+1,300); cal_h(a,n);
		ans = 0;
		cal(1,2); cal(2,1);
		printf("%I64d\n",ans);
	}
	return 0;
}
\end{lstlisting}
\subsection{后缀数组+Mobius}
\begin{lstlisting}[language=C++]
/* @hdu5219  后缀数组+Mobius
串S中不能表示为循环串的子串个数
循环串->切分成两个或以上一样的串
*/

const int N = 101100;
namespace Double{
	int t[N] , wa[N] , wb[N] , sa[N] , h[N];
	void sort(int *x,int *y,int n,int m){
		rep(i,0,m) t[i] = 0;
		rep(i,0,n) t[x[y[i]]]++;
		rep(i,1,m) t[i] += t[i-1];
		per(i,0,n) sa[--t[x[y[i]]]] = y[i];
	}
	bool cmp(int *x,int a,int b,int d){
		return x[a] == x[b] && x[a+d] == x[b+d];
	}
	void da(int *s,int n,int m){
		int *x=wa,*y=wb;
		rep(i,0,n) x[i] = s[i] , y[i] = i;
		sort(x,y,n,m);
		for(int j=1,p=1;p<n;m=p,j<<=1){
			p = 0;rep(i,n-j,n) y[p++] = i;
			rep(i,0,n) if(sa[i] >= j) y[p++] = sa[i] - j;
			sort(x,y,n,m);
			swap(x,y);p=1;x[sa[0]] = 0;
			rep(i,1,n) x[sa[i]] = cmp(y,sa[i],sa[i-1],j)?p-1:p++;
		}
	}
	void cal_h(int *s,int n,int *rk){
		int j,k=0;
		for(int i=1;i<=n;++i) rk[sa[i]] = i;
		for(int i=0;i<n;h[rk[i++]] = k)
			for(k&&--k,j=sa[rk[i]-1];s[i+k]==s[j+k];++k);
	}
}
int Log[N] , in[N] , n;
struct DA{
	int p[20][N] , rk[N];
	void Build(){
		Double::da(in,n+1,300);
		Double::cal_h(in,n,rk);
		for(int i=1;i<=n;++i) p[0][i] = Double::h[i];
		for(int j=1;1<<j<=n;++j){
			int lim = n+1-(1<<j);
			for(int i=1;i<=lim;++i)
				p[j][i] = min(p[j-1][i] , p[j-1][i+(1<<j>>1)]);
		}
	}
	int lcp(int a,int b){
		a = rk[a] , b = rk[b];
		if(a > b) swap(a,b);++a;
		int t = Log[b-a+1];
		return min(p[t][a] , p[t][b-(1<<t)+1]);
	}
}D1,D2;

int T,mu[N];
char s[N];
void wk(){
	scanf("%s",s);
	n = strlen(s);
	rep(i,0,n) in[i] = s[i]; in[n] = 0;
	D1.Build();
	rep(i,0,n) in[i] = s[n-i-1];
	D2.Build();
	ll ans = (ll)n * (n+1) / 2;
	for(int l=1;l<<1<=n;++l)
		for(int c=1;(c+1)*l<=n;++c){
			int left_len = D2.lcp(n-(c+1)*l,n-c*l);
			int right_len = (c+1)*l < n ? D1.lcp((c+1)*l,c*l) : 0;
			int tot_len = l + left_len + right_len;
			int repeat = tot_len / l;
			for(int i=2;i<=repeat;++i)
				ans += mu[i] * (tot_len - l*i + 1);
			c += right_len / l;
		}
		printf("%I64d\n",ans);
}

int main(){
	mu[1] = 1;rep(i,1,N) for(int j=i+i;j<N;j+=i) mu[j]-=mu[i];
	rep(i,2,N) Log[i] = Log[i-1] + ((i&-i) == i);
	scanf("%d",&T);
	rep(i,0,T) wk();
	return 0;
}
\end{lstlisting}
\subsection{后缀自动机}
\begin{lstlisting}[language=C++]
/* 后缀自动机 */
//最长公共子串
const int N = 250025 * 2;
int L , rt , last;
int l[N] , par[N] , ch[N][26];
void extend(int c){
    int p = last , np = ++L;
    last = np;
    l[np] = l[p] + 1;
    while(p && ch[p][c] == 0)
        ch[p][c] = np , p = par[p];
    if(!p) par[np] = rt;
    else{
        int q = ch[p][c];
        if(l[p] + 1 == l[q]) par[np] = q;
        else{
            int nq = ++L;
            l[nq] = l[p] + 1;
            memcpy(ch[nq],ch[q],sizeof ch[nq]);
            par[nq] = par[q];
            par[q] = par[np] = nq;
            while(p && ch[p][c] == q)
                ch[p][c] = nq , p = par[p];
        }
    }
}

int ans , cur , tmp;
void go(int c){
    if(ch[cur][c])
        cur = ch[cur][c] , ++tmp;
    else{
        while(cur && ch[cur][c] == 0)
            cur = par[cur];
        if(cur) tmp = l[cur] + 1 , cur = ch[cur][c];
        else cur = rt , tmp = 0;
    }
    ans = max(ans , tmp);
}
char s[N] , t[N];
int main(){
    scanf("%s",s + 1);
    scanf("%s",t + 1);
    int len1(strlen(s + 1)) , len2(strlen(t + 1));
    last = rt = ++L;
    rep(i,1,len1+1) extend(s[i] - 'a');
    tmp = ans = 0 ; cur = rt;
    rep(i,1,len2+1) go(t[i] - 'a');
    printf("%d\n",ans);
    return 0;
}
//n串最长公共子串
const int N = 101010 * 2;
int l[N] , par[N] , ch[N][26];
int last , rt , L;
void extend(int c){
    int p = last , np = ++L;
    last = np;
    l[np] = l[p] + 1;
    while(p && ch[p][c] == 0)
        ch[p][c] = np , p = par[p];
    if(!p) par[np] = rt;
    else{
        int q = ch[p][c];
        if(l[p] + 1 == l[q]) par[np] = q;
        else{
            int nq = ++L;
            l[nq] = l[p] + 1;
            memcpy(ch[nq] , ch[q] , sizeof(ch[q]));
            par[nq] = par[q];
            par[q] = par[np] = nq;
            while(p && ch[p][c] == q)
                ch[p][c] = nq , p = par[p];
        }
    }
}
int cur , step;
int si[N] , ti[N] , b[N] , t[N];
void go(int c){
    if(ch[cur][c])
        cur = ch[cur][c] , ++step;
    else{
        while(cur && ch[cur][c] == 0)
            cur = par[cur];
        if(cur) step = l[cur] + 1 , cur = ch[cur][c];
        else cur = rt , step = 0;
    }
    si[cur] = max(si[cur],step);
}
char s[11][N];
int len[11];
int _;
int main(){
    for(_=0;~scanf("%s",s[_] + 1);++_);
    rep(i,0,_) len[i] = strlen(s[i] + 1);
    rt = last = ++L;
    rep(i,1,len[0]+1) extend(s[0][i] - 'a');
    rep(i,1,L+1) ti[i] = l[i];
    rep(i,1,L+1) ++b[l[i]];
    rep(i,1,len[0]+1) b[i] += b[i-1];
    rep(i,1,L+1) t[b[l[i]]--] = i;
    rep(i,1,_){
        cur = rt , step = 0;
        rep(j,1,len[i]+1) go(s[i][j] - 'a');
        per(j,1,L+1){
            int k = t[j];
            ti[k] = min(ti[k] , si[k]);
            si[par[k]] = max(si[par[k]] , si[k]);
            si[k] = 0;
        }
    }
    int ans = 0;
    rep(i,1,L+1) ans = max(ans,ti[i]);
    printf("%d\n",ans);
    return 0;
}
//S串中长度小于等于l中出现次数最多的串
const int N = 250025 * 2;
int rt , L , last;
int l[N] , par[N] , r[N] , ch[N][26];
void extend(int c){
    int p = last , np = ++L;
    last = np;
    l[np] = l[p] + 1;
    while(p && ch[p][c] == 0)
        ch[p][c] = np , p = par[p];
    if(!p) par[np] = rt;
    else{
        int q = ch[p][c];
        if(l[p] + 1 == l[q])
            par[np] = q;
        else{
            int nq = ++L;
            l[nq] = l[p] + 1;
            memcpy(ch[nq],ch[q],sizeof ch[nq]);
            par[nq] = par[q];
            par[np] = par[q] = nq;
            while(p && ch[p][c] == q)
                ch[p][c] = nq , p = par[p];
        }
    }
}
char s[N];
int a[N] , b[N] , t[N] , ans[N];
int main(){
    scanf("%s",s + 1);
    int len(strlen(s+1));
    rt = last = ++L;
    rep(i,1,len+1) extend(a[i] = s[i] - 'a');
    rep(i,1,L+1) ++b[l[i]];
    rep(i,1,len+1) b[i] += b[i-1];
    rep(i,1,L+1) t[b[l[i]]--] = i;
    int p = rt;
    rep(i,1,len+1) p = ch[p][a[i]] , ++r[p];
    per(i,1,L+1) r[par[t[i]]] += r[t[i]];
    rep(i,1,L+1) ans[l[i]] = max(ans[l[i]] , r[i]);
    per(i,1,L+1) ans[i] = max(ans[i] , ans[i+1]);
    rep(i,1,len+1) printf("%d\n",ans[i]);
    return 0;
}
// 第K小子串
const int N = 90009 * 2;
int l[N] , par[N] , ch[N][26];
int rt , last , L;
void extend(int c){
    int p = last  , np = ++L;
    last = np;
    l[np] = l[p] + 1;
    while(p && ch[p][c] == 0)
        ch[p][c] = np , p = par[p];
    if(!p) par[np] = rt;
    else{
        int q = ch[p][c];
        if(l[p] + 1 == l[q]) par[np] = q;
        else{
            int nq = ++L;
            l[nq] = l[p] + 1;
            memcpy(ch[nq] , ch[q] , sizeof(ch[nq]));
            par[nq] = par[q];
            par[q] = par[np] = nq;
            while(p && ch[p][c] == q)
                ch[p][c] = nq , p = par[p];
        }
    }
}
char s[N];
int b[N] , t[N];
int sz[N];
char ans[N];
void Query(int k){
    int len = 0;
    int cur = rt;
    while(k){
        rep(i,0,26) if(ch[cur][i]){
            if(sz[ch[cur][i]] >= k){
                ans[len++] = 'a' + i;
                --k;
                cur = ch[cur][i];
                break;
            }
            else k -= sz[ch[cur][i]];
        }
    }
    ans[len] = 0;
    printf("%s\n",ans);
}
int main(){
    scanf("%s",s);
    int len = strlen(s);
    rt = last = ++L;
    rep(i,0,len) extend(s[i] - 'a');
    rep(i,1,L+1) b[l[i]]++;
    rep(i,0,len) b[i+1] += b[i];
    rep(i,1,L+1) t[b[l[i]]--] = i;
    per(i,1,L+1){
        int k = t[i];
        sz[k] = 1;
        rep(j,0,26) if(ch[k][j])
            sz[k] += sz[ch[k][j]];
    }


    int Q;
    for(scanf("%d",&Q);Q--;){
        int po;
        scanf("%d",&po);
        Query(po);
    }
    return 0;
}
//	S串中不在其他串中出现的子串个数
const int N = 101010 * 2;
int l[N] , par[N] , ch[N][26];
int rt , last , L;
void extend(int c){
    int p = last , np = ++L;
    Set(ch[np],0);
    last = np; l[np] = l[p] + 1;
    while(p && ch[p][c] == 0)
        ch[p][c] = np , p = par[p];
    if(!p) par[np] = rt;
    else{
        int q = ch[p][c];
        if(l[p] + 1 == l[q]) par[np] = q;
        else{
            int nq = ++L;
            Set(ch[nq],0);
            l[nq] = l[p] + 1;
            memcpy(ch[nq] , ch[q] , sizeof(ch[nq]));
            par[nq] = par[q];
            par[q] = par[np] = nq;
            while(p && ch[p][c] == q)
                ch[p][c] = nq , p = par[p];
        }
    }
}
int T,po;
ll ans;
char s[N];
int len;
int cur , step , si[N];
void go(int c){
    if(ch[cur][c])
        cur = ch[cur][c] , ++step;
    else{
        while(cur && ch[cur][c] == 0)
            cur = par[cur];
        if(cur) step = l[cur] + 1 , cur = ch[cur][c];
        else step = 0 , cur = rt;
    }
    si[cur] = max(si[cur] , step);
}
int b[N] , t[N];
void getans(){
    rep(i,0,len+1) b[i] = 0;
    rep(i,1,L+1) b[l[i]]++;
    rep(i,0,len) b[i+1] += b[i];
    rep(i,1,L+1) t[b[l[i]]--] = i;
    per(i,1,L+1){
        int k = t[i];
        si[k] = min(si[k],l[k]);
        si[par[k]] = max(si[par[k]],si[k]);
    }
    rep(i,1,L+1) si[i] = max(si[i] , l[par[i]]);
    ans = 0;
    rep(i,1,L+1) ans += l[i] - si[i];
}
int main(){
    scanf("%d",&T);
    rep(re,1,T+1){
        L = 0; rt = last = ++L;
        Set(ch[rt],0);
        scanf("%d",&po);
        scanf("%s",s);
        len = strlen(s);
        rep(i,0,len) extend(s[i] - 'a');
        rep(i,1,L+1) si[i] = 0;
        rep(cs,0,po){
            scanf("%s",s);
            cur = rt , step = 0;
            for(int i=0;s[i];++i)
                go(s[i] - 'a');
        }
        getans();
        printf("Case %d: %lld\n",re,ans);
    }
    return 0;
}
//	长度不小于k的公共子串个数
const int N = 101010 * 2;
char s[N];
int trans(char c) {
    if ('a' <= c && c <= 'z')
        return c - 'a';
    else return c - 'A' + 26;
}
int b[N], t[N];
int L;
int l[N], par[N], ch[N][52], r[N];
int rt, last;
int len;
void extend(int c) {
    int p = last, np = ++L;
    Set(ch[np], 0);
    last = np, l[np] = l[p] + 1;
    while (p && ch[p][c] == 0)
        ch[p][c] = np, p = par[p];
    if (!p) par[np] = rt;
    else {
        int q = ch[p][c];
        if (l[p] + 1 == l[q]) par[np] = q;
        else {
            int nq = ++L;
            l[nq] = l[p] + 1;
            memcpy(ch[nq], ch[q], sizeof(ch[nq]));
            par[nq] = par[q];
            par[q] = par[np] = nq;
            while (p && ch[p][c] == q)
                ch[p][c] = nq, p = par[p];
        }
    }
}
void get_right() {
    rep(i,1,L+1) r[i] = 0;
    int p = rt;
    rep(i,0,len) p = ch[p][trans(s[i])], ++r[p];
    rep(i,0,len+1) b[i] = 0;
    rep(i,1,L+1) b[l[i]]++;
    rep(i,0,len) b[i + 1] += b[i];
    rep(i,1,L+1) t[b[l[i]]--] = i;
    per(i,1,L+1){
        int k = t[i];
        r[par[k]] += r[k];
    }
}
void ini() {
    L = 0;
    rt = last = ++L;
    Set(ch[rt],0);
    scanf("%s", s);
    len = strlen(s);
    rep(i,0,len) extend(trans(s[i]));
    get_right();
}
int up , cur , step;
int g[N];
ll ans[N] , d[N];
void go(int c){
    if(ch[cur][c])
        cur = ch[cur][c] , step++;
    else{
        while(cur && ch[cur][c] == 0)
            cur = par[cur];
        if(cur) step = l[cur] + 1 , cur = ch[cur][c];
        else step = 0 , cur = rt;
    }
    d[l[par[cur]] + 1] += r[cur];
    d[step + 1] -= r[cur];
    g[par[cur]]++;
}
void cal(){
    up = min(up,len);
    per(i,1,L+1){
        int k = t[i];
        d[l[par[k]] + 1] += 1ll * g[k] * r[k];
        d[l[k] + 1] -= 1ll * g[k] * r[k];
        g[par[k]] += g[k];
    }
    rep(i,1,up+1) ans[i] = ans[i-1] + d[i];
    per(i,1,up) ans[i] += ans[i+1];
}

int main() {
    int K;
    while (~scanf("%d", &K)) {
        if (K == 0) return 0;
        ini();
        scanf("%s", s);
        cur = rt , step = 0;
        rep(i,1,L+1) g[i] = 0;
        rep(i,0,len+1) ans[i] = d[i] = 0;
        up = 0;
        for(int i=0;s[i];++i) go(trans(s[i])) , ++up;
        cal();
        printf("%lld\n", ans[K]);
    }
    return 0;
}
//	不重叠出现两次子串个数
const int inf = ~0U>>1;
const int N = 1010 * 2;
int l[N] , par[N] , ch[N][26];
int root , last , L;
void extend(int c){
    int p = last , np = ++L; Set(ch[np],0);
    last = np , l[np] = l[p] + 1;
    while(p && ch[p][c] == 0)
        ch[p][c] = np , p = par[p];
    if(!p) par[np] = root;
    else{
        int q = ch[p][c];
        if(l[p] + 1 == l[q]) par[np] = q;
        else{
            int nq = ++L;
            l[nq] = l[p] + 1;
            memcpy(ch[nq] , ch[q] , sizeof(ch[nq]));
            par[nq] = par[q];
            par[q] = par[np] = nq;
            while(p && ch[p][c] == q)
                ch[p][c] = nq , p = par[p];
        }
    }
}

char s[N];
int len;
int ans;

int lt[N] , rt[N];
int b[N] , t[N];
void cal(){
    ans = 0;
    rep(i,1,L+1) lt[i] = inf , rt[i] = -inf;
    int p = root;
    rep(i,0,len) p = ch[p][s[i] - 'a'] , lt[p] = rt[p] = i;
    rep(i,0,len+1) b[i] = 0;
    rep(i,1,L+1) b[l[i]]++;
    rep(i,0,len) b[i+1] += b[i];
    rep(i,1,L+1) t[b[l[i]]--] = i;
    per(i,1,L+1){
        int k = t[i];
        ans += min(rt[k] - lt[k] , l[k]) - min(rt[k] - lt[k] , l[par[k]]);
        lt[par[k]] = min(lt[par[k]] , lt[k]);
        rt[par[k]] = max(rt[par[k]] , rt[k]);
    }
}


int main(){
    while(~scanf("%s",s)){
        len = strlen(s);
        if(len == 1 && s[0] == '#')
            return 0;
        root = last = L = 1;
        Set(ch[1],0);
        rep(i,0,len) extend(s[i] - 'a');
        cal();
        printf("%d\n",ans);
    }

    return 0;
}
//	子串变为数字,求数字和
const int N = 101010 * 2;
int par[N] , l[N] , ch[N][10];
int root , last , L;
void extend(int c){
    int p = last , np = ++L; Set(ch[np],0);
    l[np] = l[p] + 1;
    last = np;
    while(p && ch[p][c] == 0)
        ch[p][c] = np , p = par[p];
    if(!p) par[np] = root;
    else{
        int q = ch[p][c];
        if(l[q] == l[p] + 1) par[np] = q;
        else{
            int nq = ++L;
            l[nq] = l[p] + 1;
            memcpy(ch[nq] , ch[q] , sizeof ch[nq]);
            par[nq] = par[q];
            par[q] = par[np] = nq;
            while(p && ch[p][c] == q)
                ch[p][c] = nq , p = par[p];
        }
    }
}

int n;
char s[N];
int len;
int maxlen;

void add_string(){
    scanf("%s",s);
    len = strlen(s);
    maxlen = max(maxlen , len);
    last = root;
    rep(i,0,len){
        if(ch[last][s[i] - '0'] == 0 || l[ch[last][s[i] - '0']] != i + 1)
            extend(s[i] - '0');
        else last = ch[last][s[i] - '0'];
    }
}

const int mod = 2012;
inline void Update(int &x,int y){
    x += y;
    if(x >= mod) x-=mod;
}

int ans;
int b[N] , t[N];
int way[N] , sum[N];
void get_ans(){
    ans = 0;
    rep(i,0,maxlen+1) b[i] = 0;
    rep(i,1,L+1) b[l[i]]++;
    rep(i,0,maxlen) b[i+1] += b[i];
    rep(i,1,L+1) t[b[l[i]]--] = i;
    rep(i,1,L+1) way[i] = sum[i] = 0;
    way[1] = 1;
    rep(i,1,L+1){
        int p = t[i] , q;
        Update(ans,sum[p]);
        rep(c,0,10){
            if(i == 1 && c == 0) continue;
            if(ch[p][c] == 0) continue;
            q = ch[p][c];
            Update(way[q],way[p]);
            Update(sum[q],(sum[p] * 10 + c * way[p]) % mod);
        }
    }
    printf("%d\n",ans);
}

int main(){
    while(~scanf("%d",&n)){
        if(n == 0) return 0;
        root = last = L = 1;
        Set(ch[1],0);
        maxlen = 0;
        rep(i,0,n) add_string();
        get_ans();
    }
    return 0;
}
\end{lstlisting}
\subsection{商品推荐走马灯}
\begin{lstlisting}[language=C++]
#include <iostream>
#include <cstring>
#include <cstdlib>
#include <cstdio>
#include <cmath>
#include <ctime>
#include <queue>
#include <set>
#include <map>
#include <string>
#include <bitset>
#include <vector>
#include <complex>
#include <algorithm>
using namespace std;
typedef long long ll;
typedef double db;
typedef pair<int,int> pii;
typedef vector<int> vi;
#define de(x) cout << #x << "=" << x << endl
#define rep(i,a,b) for(int i=a;i<(b);++i)
#define all(x) (x).begin(),(x).end()
#define sz(x) (int)(x).size()
#define mp make_pair
#define pb push_back
#define fi first
#define se second
const int N = 100005;
int s[N];int l[N<<1] , n , m;

struct FastIO {
    static const int S = 1310720;
    int wpos; char wbuf[S];
    FastIO() : wpos(0) {}
    inline int xchar() {
        static char buf[S];
        static int len = 0, pos = 0;
        if (pos == len)
            pos = 0, len = fread(buf, 1, S, stdin);
        if (pos == len) return -1;
        return buf[pos ++];
    }
    inline int xuint() {
        int c = xchar(), x = 0;
        while (c <= 32 && ~c) c = xchar();
        if (c == -1) return -1;
        for (;'0' <= c && c <= '9'; c = xchar()) x = x * 10 + c - '0';
        return x;
    }
    inline int xint() {
        int s = 1, c = xchar(), x = 0;
        while (c <= 32) c = xchar();
        if (c == '-') s = -1, c = xchar();
        for (; '0' <= c && c <= '9'; c = xchar()) x = x * 10 + c - '0';
        return x * s;
    }
    inline void xstring(char *s) {
        int c = xchar();
        while (c <= 32) c = xchar();
        for(; c > 32; c = xchar()) *s++ = c;
        *s = 0;
    }
    inline void wchar(int x) {
        if (wpos == S) fwrite(wbuf, 1, S, stdout), wpos = 0;
        wbuf[wpos ++] = x;
    }
    inline void wint(ll x) {
        if (x < 0) wchar('-'), x = -x;

        char s[24];
        int n = 0;
        while (x || !n) s[n ++] = '0' + x % 10, x /= 10;
        while (n--) wchar(s[n]);
    }
    inline void wstring(const char *s) {
        while (*s) wchar(*s++);
    }
    ~FastIO() {
        if (wpos) fwrite(wbuf, 1, wpos, stdout), wpos = 0;
    }
} io;
void manacher(int *s,int n,int *l){
    l[0]=1;
    for(int i=1,j=0;i<(n<<1)-1;++i){
        int p=i>>1,q=i-p,r=((j+1)>>1)+l[j]-1;
        l[i]=r<q?0:min(r-q+1,l[(j<<1)-i]);
        while(p-l[i]!=-1&&q+l[i]!=n&&s[p-l[i]]==s[q+l[i]]) l[i]++;
        if(q+l[i]-1>r) j=i;
    }
}

vector<pair<pii,int> > ql,qr;
ll sum[N],ans[N];
struct Tree{
#define lson c<<1,l,m
#define rson c<<1|1,m+1,r
#define pos int c,int l,int r
    ll sum[N<<2],f[N<<2],g[N<<2],ans[N<<2],len[N<<2];
    void up(int c){ans[c]=ans[c<<1]+ans[c<<1|1];}
    void F(int c,ll t){f[c]+=t,ans[c]+=t*sum[c];}
    void G(int c,ll t){g[c]+=t,ans[c]+=t*len[c];}
    void down(int c){
        if(f[c]) F(c<<1,f[c]) , F(c<<1|1,f[c]) , f[c]=0;
        if(g[c]) G(c<<1,g[c]) , G(c<<1|1,g[c]) , g[c]=0;
    }
    void Build(pos){
        sum[c]=f[c]=g[c]=ans[c]=0;len[c]=r-l+1;
        if(l==r) sum[c]=(l?::sum[l-1]:0);
        else {
            int m=(l+r)>>1;
            Build(lson);Build(rson);
            sum[c]=sum[c<<1]+sum[c<<1|1];
        }
    }
    ll Ans(pos,int L,int R){
        if(L <= l && r <= R) return ans[c];
        else {
            int m=(l+r)>>1;ll s=0;
            down(c);
            if(L<=m) s+=Ans(lson,L,R);
            if(R>m) s+=Ans(rson,L,R);
            return s;
        }
    }
    void Mod(pos,int L,int R,ll x,int type){
        if(L <= l && r <= R) type?G(c,x):F(c,x);
        else {
            int m=(l+r)>>1;
            down(c);
            if(L<=m) Mod(lson,L,R,x,type);
            if(R>m) Mod(rson,L,R,x,type);
            up(c);
        }
    }
}tree;

void solve(int *s,vector<pair<pii,int> >&v){
    manacher(s,n,l);
    sort(all(v));
    rep(i,0,n) sum[i]=(i?sum[i-1]:0)+s[i];
#define rt 1,0,n-1
    tree.Build(rt);
    for(int i=0,j=0;i<(n<<1)-1;++i){
        int p=i>>1;
        if(l[i]){
            tree.Mod(rt,p-l[i]+1,p,-2,0); // *
            tree.Mod(rt,p-l[i]+1,p,sum[p]+(i&1?sum[p]:(i?sum[p-1]:0)),1); // +
        }
        for(;j<sz(v)&&v[j].fi.fi==i;++j)
            ans[v[j].se]+=tree.Ans(rt,v[j].fi.se,p);
    }
}

int main(){
    n = io.xint();
    m = io.xint();
    rep(i,0,n) s[i] = io.xint();
    rep(i,0,m){
        int u , v;
        u = io.xint() , v = io.xint();
        --u;--v;
        if(u > v) swap(u,v);
        ql.pb(mp(mp(u+v,u),i));
        if(u==v) continue;
        u=n-1-u,v=n-1-v;swap(u,v);
        qr.pb(mp(mp(u+v-1,u),i));
    }
    solve(s,ql);
    reverse(s,s+n);
    solve(s,qr);
    rep(i,0,m) io.wint(ans[i]) , io.wchar('\n');
    return 0;
}
\end{lstlisting}
\subsection{平面图}
\begin{lstlisting}[language=C++]
#include <iostream>
#include <cstring>
#include <cstdlib>
#include <cstdio>
#include <cmath>
#include <ctime>
#include <queue>
#include <set>
#include <map>
#include <string>
#include <bitset>
#include <vector>
#include <complex>
#include <algorithm>
using namespace std;
typedef long long ll;
typedef double db;
typedef pair<int,int> pii;
typedef vector<int> vi;
#define de(x) cout << #x << "=" << x << endl
#define rep(i,a,b) for(int i=a;i<(b);++i)
#define all(x) (x).begin(),(x).end()
#define sz(x) (int)(x).size()
#define mp make_pair
#define pb push_back
#define fi first
#define se second
const int inf = 1e9 + 10;
typedef db T;
typedef complex<T> P;
T dot(P a,P b){return (conj(a)*b).real();}
T cross(P a,P b){return (conj(a)*b).imag();}
T angle(P a,P b){return arg(b-a);}
T angle(P a,P o,P b){return abs(remainder(arg(a-o)-arg(b-o),2.*M_PI));}
P in(){
    db x,y;scanf("%lf%lf",&x,&y);
    return P(x,y);
}
const int N = 10100 * 2;
P a[N],b[N];
vi g[N],cir[N];db area[N];int id[N] , ccnt;
struct L{
    int s,t,c,v,nxt;
    L(int _s=0,int _t=0,int _c=0,int _v=0){s=_s,t=_t,c=_c,v=_v;}
}l[N];int lcnt;
bool Cmp(int x,int y){
    return angle(b[l[x].s],b[l[x].t]) < angle(b[l[y].s],b[l[y].t]);
}
void ins(int u,int v,int c){
    l[lcnt] = L(u,v,c,0);
    g[u].pb(lcnt++);
}
int cnta,n,m,u,v,c;
bool in(P a,vi&cir){
    db ag = 0.;
    rep(i,0,sz(cir)){
        P u = b[l[cir[i]].s] , v = b[l[cir[i]].t];
        db del = angle(u , a , v);
        if(cross(u - a , v - a) >= 1e-8) ag += del;
        else ag -= del;
    }
    return fabs(ag) > 1;
}
bool in(vi&cira,vi&cir){
    rep(j,0,sz(cira)) if(!in(b[l[cira[j]].s] , cir)) return false;
    return true;
}

template<class T>
struct Dinic{ // [0,n) init!!
    const static int N = 200 , M = 1000;
    int s , t , n , h[N] , cur[N] , level[N] , q[N] , e , ne[M] , to[M];
    T cap[M] , flow;
    void liu(int u,int v,T w){ to[e] = v;ne[e] = h[u];cap[e] = w;h[u] = e++;};
    void link(int u,int v,T w){ liu(u , v , w);liu(v , u , 0);}
    void ini(int _n = N) { fill(h , h + (n=_n) , -1);e = 0;}
    bool bfs(){
        int L = 0 , R = 0;
        fill(level , level + n , -1);
        level[q[R++] = s] = 0;
        while(L < R && level[t] == -1){
            int c = q[L++];
            for(int k=h[c];~k;k=ne[k])
                if(cap[k] > 0 && level[to[k]] == -1)
                    level[q[R++] = to[k]] = level[c] + 1;
        }
        return ~level[t];
    }
    T dfs(int c,T mx){
        if(c == t) return mx;
        T ret = 0;
        for(int &k = cur[c];~k;k = ne[k]){
            if(level[to[k]] == level[c] + 1 && cap[k] > 0){
                T flow = dfs(to[k] , min(mx , cap[k]));
                ret += flow;cap[k] -= flow , cap[k^1] += flow;mx -= flow;
                if(!mx) return ret;
            }
        }
        level[c] = -1;
        return ret;
    }
    T run(int _s,int _t){
        s = _s , t = _t;
        while(bfs()){
            copy(h , h + n , cur);
            flow += dfs(s,~0U>>1);
        }
        return flow;
    }
};
Dinic<int> flow[11];
int ans[100] , belong[100];
int SS , TT;
void dfs(int dep,int mask){
    flow[dep].run(SS,TT);
    if(dep == cnta){
        int c1 = __builtin_popcount(mask);
        ans[c1] = min(ans[c1] , flow[dep].flow);
    }
    else{
        flow[dep+1] = flow[dep];
        dfs(dep + 1 , mask << 1);
        flow[dep+1].link(SS , belong[dep] , inf);
        dfs(dep + 1 , mask << 1 | 1);
    }
}

int main(){
    fill(ans , ans + 100 , inf);
    scanf("%d%d%d",&cnta,&n,&m);
    rep(i,0,cnta) a[i] = in();
    rep(i,0,n) b[i] = in();
    b[n] = P(-inf,-inf) , b[n + 1] = P(inf,-inf);
    b[n + 2] = P(inf,inf) , b[n + 3] = P(-inf,inf);
    rep(i,0,m){
        scanf("%d%d%d",&u,&v,&c);
        --u;--v;
        ins(u , v , c);
        ins(v , u , c);
    }
    rep(i,0,4) ins(n + i , n + (i+1)%4 , inf) , ins(n + (i+1)%4 , n + i , inf);
    rep(i,0,n+4){
        sort(all(g[i]) , Cmp);
        rep(j,0,sz(g[i])){
            int u = g[i][j]^1 , v = g[i][(j+1)%sz(g[i])];
            l[u].nxt = v;
        }
    }
    rep(i,0,lcnt) if(!l[i].v){
        ++ccnt;int c = i , s = l[i].s;
        area[ccnt] = 0.;
        do {l[c].v = ccnt , cir[ccnt].pb(c) , area[ccnt] += cross(b[l[c].s],b[l[c].t]) , c = l[c].nxt;}while(l[c].s != s);
        //rep(i,0,sz(cir[ccnt])) cout << cir[ccnt][i] << " " << b[l[cir[ccnt][i]].s] << " \n"[i == sz(cir[ccnt]) - 1];
        //cout << area[ccnt] << endl;
    }
    // area < 0 : in
    db Min = 0;
    SS = 0 , TT = 0;
    rep(i,1,ccnt+1) if(area[i] < Min) Min = area[i] , TT = i;
    rep(i,1,ccnt+1)
        if(area[i] >= 0){
            rep(j,1,ccnt+1) if(area[j] < 0 && (fabs(area[j]) > fabs(area[i])))
                if(in(cir[i], cir[j]))
                    if(!id[i] || (area[j] > area[id[i]]))
                        id[i] = j;
        }
    rep(i,1,ccnt+1) if(!id[i]) id[i] = i;
    rep(i,0,cnta) rep(j,1,ccnt+1) if(area[j] < 0 && in(a[i] , cir[j]))
        if(!belong[i] || area[j] > area[belong[i]]) belong[i] = j;
    flow[0].ini(ccnt + 1);
    rep(i,0,lcnt) flow[0].liu(id[l[i].v] , id[l[i^1].v] , l[i].c);
    dfs(0,0);
    rep(i,1,cnta+1) printf("%d\n",ans[i]);
    return 0;
}
\end{lstlisting}
\subsection{平面生成树构造}
\begin{lstlisting}[language=C++]
/*	@Gym-100430I
	平面点集度数已知,构造生成树,线不相交
*/

const int N = 1001;
int x[N] , y[N] , d[N] , n;
vector<pii> ans;
db ang[N];
bool cmp(const int a,const int b){
	return ang[a] < ang[b];
}
void solve(int,int,db,vi&);
void link(int s,vi& v){
	if(sz(v) == 0) return;
	int sum = 0 , o = 0 , d1 = 0;
	for(;o<sz(v);++o){
		int a = v[o];
		sum += d[a] - 2;
		if(sum >= 0 || o + 1 == sz(v)){
			sum -= d[a] - 2;
			d1 = -sum;
			break;
		}
	}
	vi nv , mv;
	rep(i,0,o) nv.pb(v[i]);
	rep(i,o+1,sz(v)) mv.pb(v[i]);
	o = v[o];
	ans.pb(mp(o,s));
	db hori = atan2(y[o] - y[s] , x[o] - x[s]);
	solve(o,d1,hori,nv);
	solve(o,d[o]-1-d1,hori,mv);
}
void solve(int s,int du,db horizon,vi& v){
	if(sz(v) == 0) return;
	rep(i,0,sz(v)){
		int a = v[i];
		db tmp = atan2(y[a] - y[s] , x[a] - x[s]) - horizon;
		while(tmp <= -pi) tmp += 2*pi;
		while(tmp > pi) tmp -= 2*pi;
		ang[a] = tmp;
	}
	sort(all(v),cmp);
	vi nv;
	int sum = 0 , psum = 0;
	rep(i,0,sz(v)){
		int a = v[i];
		sum += d[a] - 2;
		nv.pb(a);
		if((sum < psum && sum + du >= 1) || i + 1 == sz(v)){
			psum = sum;
			link(s,nv);
			nv.clear();
		}
	}
}

int main(){
	scanf("%d",&n);
	rep(i,0,n) scanf("%d%d%d",x + i , y + i , d + i);
	int start = min_element(x , x + n) - x;
	vi v; rep(i,0,n) if(i != start) v.pb(i);
	solve(start,d[start],0.,v);
	rep(i,0,n-1) printf("%d %d\n",ans[i].fi + 1,ans[i].se + 1);
	return 0;
}
\end{lstlisting}
\subsection{微软项目经理的挑选方案}
\begin{lstlisting}[language=C++]
#include <algorithm>
#include <iostream>
#include <cstring>
#include <vector>
#include <cstdio>
#include <string>
#include <cmath>
#include <queue>
#include <set>
#include <map>
using namespace std;
typedef long long ll;
typedef double db;
typedef pair<int,int> pii;
typedef vector<int> vi;
#define de(x) cout << #x << "=" << x << endl
#define rep(i,a,b) for(int i=a;i<(b);++i)
#define all(x) (x).begin(),(x).end()
#define sz(x) (int)(x).size()
#define mp make_pair
#define pb push_back
#define fi first
#define se second
#define setIO(x) freopen(x".in","r",stdin);freopen(x".out","w",stdout)
const int N = 202020 , P = 1e9 + 7;
int n;
struct ele{
    int l,r;
    bool operator < (const ele&b) const{
        if(r!=b.r) return r<b.r;
        return l<b.l;
    }
}e[N];
int L[N] , R[N];
vi V;
#define lb(x) (x&(-x))
struct Fenwick{
    int a[N<<1],l;
    int Max(int d){
        int r=0;
        for(int i=d;i>=1;i-=lb(i)) r=max(r,a[i]);
        return r;
    }
    void ins(int d,int x){
        for(int i=d;i<=l;i+=lb(i)) a[i]=max(a[i],x);
    }
}f,g;
void pp(int &a,int d){a+=d;if(a>=P)a-=P;}
int sum[N<<1] , lazy[N<<1];
#define id(l,r) (l+r)|(l!=r)
#define pr int c=id(l,r),mid=(l+r)>>1
void Down(int l,int r){pr;
    if(lazy[c] != 1){
        int lc=id(l,mid),rc=id(mid+1,r);
        lazy[lc] = (ll)lazy[lc]*lazy[c]%P;
        lazy[rc] = (ll)lazy[rc]*lazy[c]%P;
        sum[lc] = (ll)sum[lc]*lazy[c]%P;
        sum[rc] = (ll)sum[rc]*lazy[c]%P;
        lazy[c] = 1;
    }
}
void Up(int l,int r){pr;
    int lc=id(l,mid),rc=id(mid+1,r);
    sum[c] = (sum[lc]+sum[rc])%P;
}
void Ins(int l,int r,int x,int d){ pr;
    if(l == r){
        pp(sum[c] , d);
    } else{
        Down(l,r);
        if(x<=mid) Ins(l,mid,x,d);
        else Ins(mid+1,r,x,d);
        Up(l,r);
    }
}
void Mul2(int l,int r,int L,int R){pr;
    if(L <= l && r <= R){
        pp(lazy[c],lazy[c]);
        pp(sum[c],sum[c]);
    } else {
        Down(l,r);
        if(L<=mid) Mul2(l,mid,L,R);
        if(R>mid) Mul2(mid+1,r,L,R);
        Up(l,r);
    }
}
int Sum(int l,int r,int L,int R){pr;
    if(L <= l && r <= R){
        return sum[c];
    } else {
        Down(l,r);
        int ret=0;
        if(L<=mid) pp(ret,Sum(l,mid,L,R));
        if(R>mid) pp(ret,Sum(mid+1,r,L,R));
        Up(l,r);
        return ret;
    }
}


int main(){
    scanf("%d",&n);
    rep(i,1,n+1) scanf("%d%d",&e[i].l,&e[i].r) , V.pb(e[i].l) , V.pb(e[i].r);
    sort(all(V));V.erase(unique(all(V)) , V.end());
    sort(e + 1, e + n + 1);
#define Rk(x) lower_bound(all(V) , x) - V.begin() + 1
    rep(i,1,n+1) e[i].l = Rk(e[i].l) , e[i].r = Rk(e[i].r);
    f.l = g.l = sz(V) + 1;
    rep(i,1,n+1){
        L[i] = f.Max(e[i].l - 1);
        f.ins(e[i].r,i);
    }
    for(int i=n;i>=1;--i){
        R[i] = n + 1 - g.Max(f.l - e[i].r);
        g.ins(f.l - e[i].l + 1 ,n + 1 - i);
    }
    fill(lazy,lazy+(N<<1),1);
    Ins(0,n+1,1,1);Ins(0,n+1,R[1],1);
    rep(i,2,n+1){
        Ins(0,n+1,R[i],Sum(0,n+1,L[i]+1,R[i]));
        Mul2(0,n+1,0,L[i]);
    }
    printf("%d\n",Sum(0,n+1,n+1,n+1));
    return 0;
}
\end{lstlisting}
\subsection{有根树点分治}
\begin{lstlisting}[language=C++]
const int N = 201010;
int vis[N] , n , p[N] , par[N];
ll q[N] , lim[N] , dis[N] , ans[N];
vector<pair<int,ll> > g[N];
ll getans(int x,int y){return ans[y]+q[x]-p[x]*dis[y];}
void Update(int x,int y){if(dis[y] >= lim[x]) ans[x]=min(ans[x],getans(x,y));}

int getrt(int c){
    static int par[N] , sz[N] , q[N];
    int _ = 0;par[q[_++] = c] = -1;
    rep(i,0,_){
        int c = q[i];sz[c] = 1;
        for(auto e : g[c]) if(!vis[e.fi] && e.fi != par[c]) par[q[_++] = e.fi] = c;
    }
    for(int i=_-1;i>=0;--i){
        int c = q[i];
        if(sz[c]*2>_) return c;
        sz[par[c]] += sz[c];
    }return -1;
}
void dfs(int c,int rt){
    Update(c , rt);
    for(auto e : g[c]) if(!vis[e.fi] && e.fi != par[c]) dfs(e.fi , rt);
}
int e[N],_;
typedef complex<db> PP;
PP po[N];int f[N],tf;
db cross(PP a,PP b,PP c){return (conj(b-a)*(c-a)).imag();}
void Push(int c){
    e[_++] = c;
    for(auto e : g[c]) if(!vis[e.fi] && e.fi != par[c]) Push(e.fi);
}
void add(int x){
    PP tp = PP(-dis[x],ans[x]);
    while(tf > 1 && cross(po[tf-2],po[tf-1],tp) <= 0) --tf;
    f[tf] = x , po[tf++] = tp;
}
void Query(int x){
    if(tf == 0) return;
    int l=0,r=tf;
    while(l+1<r){
        int m=(l+r)>>1;
        if(getans(x,f[m]) <= getans(x,f[m-1])) l=m;
        else r=m;
    }
    Update(x,f[l]);
}

void solve(int c){
    int rt = getrt(c);
    vis[rt] = true;
    if(c != rt){
        solve(c);_ = tf = 0;
        for(auto e : g[rt]) if(!vis[e.fi]) Push(e.fi);
        sort(e , e + _ , [&](int a,int b){return lim[a] < lim[b];});
        for(int i=par[rt];i!=par[c];i=par[i]) Update(rt , i);
        for(int i=rt,j=_-1;j>=0;Query(e[j--]))
            for(;i!=par[c]&&dis[i] >= lim[e[j]];i = par[i])
                add(i);
    } else for(auto e : g[rt]) if(!vis[e.fi]) dfs(e.fi,rt);
    for(auto e : g[rt]) if(!vis[e.fi]) solve(e.fi);
}

void dfs_ini(int c){
    q[c] += dis[c] * p[c] , lim[c] = dis[c] - lim[c];
    for(auto e : g[c]) if(e.fi != par[c])
        par[e.fi] = c , dis[e.fi] = dis[c] + e.se , dfs_ini(e.fi);
}

int main(){
    int u;ll w;
    scanf("%d%*d",&n);
    rep(i,2,n+1) scanf("%d%lld%d%lld%lld",&u,&w,p + i,q + i,lim + i) , g[i].pb(mp(u,w)) , g[u].pb(mp(i,w));
    fill_n(ans+2,n-1,1ll<<62);
    dfs_ini(1);solve(1);
    rep(i,2,n+1) printf("%lld\n",ans[i]);
    return 0;
}
\end{lstlisting}
\subsection{树链剖分}
\begin{lstlisting}[language=C++]
/*	@hdu-5405
	单点修改,去链后子树sum^2和
*/
const int N = 101010 , M = N << 1;
int n , m;
#define lb(x) ((x)&(-x))
struct BIT{
	ll a[N];
	void ini(){
		fill(a+1,a+1+n,0);
	}
	void Pre(){
		for(int i=1,j=i+lb(i);i<=n;++i,j=i+lb(i)) if(j<=n) a[j] += a[i];
	}
	void add(int x,ll d){
		for(;x<=n;x+=lb(x)) a[x] += d;
	}
	ll sum(int x){
		ll r=0; for(;x>=1;x-=lb(x)) r += a[x]; return r;
	}
};
BIT bi,bj;
int to[M] , ne[M] , h[N] , e;
void link(int u,int v){
	to[e] = v , ne[e] = h[u];
	h[u] = e++;
}
int par[N] , dep[N] , sz[N] , wson[N];
int lft[N] , rht[N] , tp[N] , _;
ll s[N];
int w[N];ll wsum;
void dfs(int c,int fa){
	par[c] = fa , dep[c] = dep[fa] + 1 , sz[c] = 1 , wson[c] = 0;
	for(int k=h[c];~k;k=ne[k]){
		int t = to[k];
		if(t == fa) continue;
		dfs(t , c);
		sz[c] += sz[t];
		if(sz[t] > sz[wson[c]]) wson[c] = t;
	}
}
inline ll sqr(ll x){
	x %= mod;
	return x*x%mod;
}
void dds(int c,int pa){
	lft[c] = ++_;
	tp[c] = pa;
	bi.a[_] = w[c]; // bi
	s[c] = w[c];
	ll tmp = 0;
	if(wson[c]){
		dds(wson[c],pa);
		s[c] += s[wson[c]];
	}
	for(int k=h[c];~k;k=ne[k]){
		int t = to[k];
		if(t == par[c] || t == wson[c]) continue;
		dds(t,t);
		s[c] += s[t];
		(tmp += sqr(s[t])) %= mod;
	}
	rht[c] = _;
	bj.a[lft[c]] = tmp;
}
ll Query_s(int x){
	return bi.sum(rht[x]) - bi.sum(lft[x]-1);
}
void Change(int a,int b){
	ll del = b - w[a];
	w[a] = b;wsum += del;
	bi.add(lft[a],del); // bi
	int fa = tp[a];
	while(a = par[fa]){
		ll gt = Query_s(fa);
		bj.add(lft[a],sqr(gt) - sqr(gt-del));
		fa = tp[a];
	}
}
int Query(int a,int b){
	ll ret = 0;
	int fa = tp[a] , fb = tp[b];
	while(fa != fb){
		if(dep[fa] < dep[fb]) swap(a,b) , swap(fa,fb);
		ret += bj.sum(lft[a]) - bj.sum(lft[fa]-1);
		if(wson[a]) ret += sqr(Query_s(wson[a]));
		ret -= sqr(Query_s(fa));
		a = par[fa] , fa = tp[a];
	}
	if(dep[a] < dep[b]) swap(a,b);
	if(wson[a]) ret += sqr(Query_s(wson[a]));
	ret += bj.sum(lft[a]) - bj.sum(lft[b]-1);
	ret += sqr(wsum - Query_s(b));
	ret = sqr(wsum) - ret;
	ret %= mod , (ret += mod) %= mod;
	return (int)ret;
}
void work(){
	wsum = 0;
	rep(i,1,n+1) scanf("%d",w+i) , wsum += w[i];
	int u , v;
	e = 0;fill(h+1,h+1+n,-1);
	rep(i,1,n){
		scanf("%d%d",&u,&v);
		link(u,v);
		link(v,u);
	}
	bi.ini();bj.ini();
	dfs(1,0);_=0;dds(1,1);
	bi.Pre();bj.Pre();
	int op , a , b;
	rep(i,0,m){
		scanf("%d%d%d",&op,&a,&b);
		if(op == 1) Change(a,b);
		else printf("%d\n",Query(a,b));
	}
}
int main(){
	while(~scanf("%d%d",&n,&m)) work();
	return 0;
}
\end{lstlisting}
\subsection{百度帐号的选取方案}
\begin{lstlisting}[language=C++]
#include <algorithm>
#include <iostream>
#include <cstring>
#include <vector>
#include <cstdio>
#include <string>
#include <queue>
#include <set>
#include <map>
#include <bitset>
using namespace std;
typedef long long ll;
typedef double db;
typedef pair<int,int> pii;
typedef vector<int> vi;
#define de(x) cout << #x << "=" << x << endl
#define rep(i,a,b) for(int i=a;i<b;++i)
#define per(i,a,b) for(int i=b-1;i>=a;--i)
#define all(x) (x).begin(),(x).end()
#define sz(x) (int)(x).size()
#define mp make_pair
#define pb push_back
#define fi first
#define se second
#define setIO(x) freopen(x".in","r",stdin);freopen(x".out","w",stdout);

namespace Doubling{
    static const int N = 101010;
    int t[N] , wa[N] , wb[N] , sa[N] , h[N];
    void sort(int *x,int *y,int n,int m){
        rep(i,0,m) t[i] = 0;
        rep(i,0,n) t[x[y[i]]]++;
        rep(i,1,m) t[i] += t[i-1];
        per(i,0,n) sa[--t[x[y[i]]]] = y[i];
    }
    bool cmp(int *x,int a,int b,int d){
        return x[a] == x[b] && x[a+d] == x[b+d];
    }
    void da(int *s,int n,int m){
        int *x=wa,*y=wb;
        rep(i,0,n) x[i] = s[i] , y[i] = i;
        sort(x , y , n , m);
        for(int j=1,p=1;p<n;m=p,j<<=1){
            p = 0;rep(i,n-j,n) y[p++] = i;
            rep(i,0,n) if(sa[i] >= j) y[p++] = sa[i] - j;
            sort(x , y , n , m);
            std::swap(x , y);p = 1;x[sa[0]] = 0;
            rep(i,1,n) x[sa[i]] = cmp(y,sa[i],sa[i-1],j)?p-1:p++;
        }
    }
    void cal_h(int *s,int n,int *rk){
        int j,k=0;
        for(int i=1;i<=n;++i) rk[sa[i]] = i;
        for(int i=0;i<n;h[rk[i++]] = k)
            for(k&&--k,j=sa[rk[i]-1];s[i+k]==s[j+k];++k);
    }
}
struct DA{
    static const int N = 101010;
    int p[20][N] , rk[N] , in[N] , Log[N] , n;
    void Build(){
        Doubling::da(in,n+1,300);
        Doubling::cal_h(in,n,rk);
        Log[0] = -1;for(int i=1;i<=n;++i) Log[i] = Log[i-1] + (i==(i&(-i)));
        for(int i=1;i<=n;++i) p[0][i] = Doubling::h[i];
        for(int j=1;1<<j<=n;++j){
            int lim = n+1-(1<<j);
            for(int i=1;i<=lim;++i)
                p[j][i] = std::min(p[j-1][i] , p[j-1][i+(1<<j>>1)]);
        }
    }
    int lcp(int a,int b){
        a = rk[a] , b = rk[b];
        if(a > b) std::swap(a , b);++a;
        int t = Log[b-a+1];
        return std::min(p[t][a] , p[t][b-(1<<t)+1]);
    }
}D1,D2;
const int N = 101010;
char s[N];int n;
vector<pair<pii,int> > V[N] , E[N];
pii p[N];

const int M = N<<1;
ll lcnt[N] , rcnt[N];
ll Sum(int l,int r){
    return (ll)(l+r)*(r-l+1)>>1;
}
ll solve(vector<pair<pii,int> > pv){
    vector<pii> v;
    for(auto e : pv){
        int l = e.fi.fi , r = e.fi.se , L = e.se , R = L+r-l;
        lcnt[l]++ , lcnt[L+1]-- , rcnt[r]++ , rcnt[R+1]--;
        v.pb(mp(l-1,1));v.pb(mp(L,2)); // 2 : + , 1 : -
        v.pb(mp(r-1,-2));v.pb(mp(R,-1));
    }
    sort(all(v));
    ll sz = 0 , sum = 0 , f = 0 , pre = -1 , ret = 0;
    for(int i=0;i<sz(v);){
        int nt = v[i].fi;
        sz += (nt - pre) * f;sum += Sum(pre + 1, nt) * f;
        for(;i<sz(v) && v[i].fi == nt;++i){
            if(v[i].se < 0) f += (v[i].se == -2 ? 1 : -1);
            else ret += (v[i].se == 2 ? 1 : -1) * (nt * sz - sum);
        }
        pre = nt;
    }
    return ret;
}

ll solve1(){
    rep(i,1,n) lcnt[i] += lcnt[i-1] , rcnt[i] += rcnt[i-1];
    //rep(i,0,n) printf("%d %d\n",lcnt[i] , rcnt[i]);
    rep(i,1,n) rcnt[i] += rcnt[i-1];
    ll ret = 0;
    rep(i,1,n) ret += (Sum(1,i) - rcnt[i-1]) * (n-i - lcnt[i]);
    return ret;
}
void Upf(int &x,int d){
    if(x < d) x = d;
}
int main(){
    scanf("%s",s);
    n = strlen(s);
    D1.n = n;rep(i,0,n) D1.in[i] = s[i];D1.in[n] = 0;
    D1.Build();
    D2.n = n;rep(i,0,n) D2.in[i] = s[n-i-1];D2.in[n] = 0;
    D2.Build();
    for(int l=1;l<<1<=n;++l)
        for(int c=1;(c+1)*l<=n;++c){
            int left_cnt = D2.lcp(n-(c+1)*l , n-c*l);
            int right_cnt = (c+1)*l<n ? D1.lcp((c+1)*l , c*l) : 0;
            int tot_len = left_cnt + right_cnt + l;
            int repeat = tot_len / l;
            int L = c*l - left_cnt , R = (c+1)*l + right_cnt;
            for(int i=2;i<=repeat;++i){
                V[L].pb(mp(mp(L+i*l-1,R-i*l),i)); // r , R , x
            }
            c += right_cnt / l;
        }
    rep(i,0,n) p[i] = mp(0,0);
    rep(i,0,n){
        for(auto e : V[i]) p[e.fi.fi] = max(p[e.fi.fi] , mp(e.fi.se , e.se));
        for(auto e : V[i]){
            pii&t = p[e.fi.fi];
            if(t != mp(0,0)) E[t.se].pb(mp(mp(i , e.fi.fi) , t.fi)) , t = mp(0,0); // l , r , L
        }
    }
    ll ans = 0;
    rep(i,2,n+1) ans += solve(E[i]);
    ans += solve1();
    printf("%lld\n",ans);
    return 0;
}
\end{lstlisting}
\subsection{线段树+sg}
\begin{lstlisting}[language=C++]
/*	@hdu-5345 sg+线段树
	sg[x][y] = lowbit(max(x,y))
	矩形并异或和
*/
int n , R , C;
struct event{
	int l,r,v;
	event(int _l=0,int _r=0,int _v=0){l=_l,r=_r,v=_v;}
}e[N<<1]; int ne[N<<1] , h[N] , e_cnt;
vi w;
void add(int x,int l,int r,int v){
	e[e_cnt] = event(l,r,v);
	ne[e_cnt] = h[x];
	h[x] = e_cnt++;
	w.pb(l); w.pb(r+1);
}

int cal(int l,int r){
	if(l > r) return 0;
	int ret = 0;
	for(int i=1;i<=r;i<<=1){
		int u , v;
		u = (r - i) / (2*i) + 1;
		v = l - 1 < i ? 0 : (l - 1 - i) / (2*i) + 1;
		if(u-v&1) ret += i;
	}
	return ret;
}

#define lson get_id(l,m)
#define rson get_id(m+1,r)
int sum[N<<2] , cov[N<<2] , sz[N<<2] , sgs[N<<2];
inline int get_id(int l,int r){
	return l + r | l != r;
}
void Build(int l,int r){
	int x = get_id(l,r);
	int m = l+r>>1;
	cov[x] = sz[x] = sgs[x] = 0; //
	if(l == r){
		sum[x] = cal(w[l] , w[r+1] - 1);
		return;
	}
	Build(l,m); Build(m+1,r);
	sum[x] = sum[lson] ^ sum[rson];
}
void Cover(int l,int r,int L,int R,int v){
	int x = get_id(l,r);
	int m = l+r>>1;
	if(L <= l && r <= R){
		cov[x] += v;
		if(cov[x]){
			sz[x] = w[r+1] - w[l];
			sgs[x] = sum[x];
		}
		else if(l == r) sz[x] = sgs[x] = 0;
		else sz[x] = sz[lson] + sz[rson] , sgs[x] = sgs[lson] ^ sgs[rson];
		return;
	}
	if(L <= m) Cover(l,m,L,R,v);
	if(m < R) Cover(m+1,r,L,R,v);
	if(!cov[x]) sz[x] = sz[lson] + sz[rson] , sgs[x] = sgs[lson] ^ sgs[rson];
}
int sg , szcnt , nowx , covcnt;
void Query_sz2(int l,int r){
	int x = get_id(l,r) , m = l+r>>1;
	if(cov[x]){
		sg ^= cal(nowx,w[r+1] - 1);
		szcnt += min(nowx,w[r+1]) - w[l];
	}
	else if(l < r){
		szcnt += sz[lson];
		Query_sz2(m+1,r);
	}
}
void Query_sz(int l,int r,int L,int R){
	int x = get_id(l,r) , m = l+r>>1;
	covcnt += cov[x];
	if(L <= l && r <= R){
		if(R == r){
			if(covcnt) sg ^= cal(nowx,w[r+1]-1) , szcnt += min(nowx,w[r+1]) - w[l];
			else Query_sz2(l,r);
		}
		else if(covcnt) szcnt += w[r+1] - w[l];
		else szcnt += sz[x];
		return ;
	}
	int cc = covcnt;
	if(L <= m) Query_sz(l,m,L,R); covcnt = cc;
	if(m < R) Query_sz(m+1,r,L,R);
}
void Query_sgs(int l,int r,int L,int R){
	int x = get_id(l,r) , m = l+r>>1;
	covcnt += cov[x];
	if(L <= l && r <= R){
		if(covcnt) sg ^= sum[x];
		else sg ^= sgs[x];
		return;
	}
	int cc = covcnt;
	if(L <= m) Query_sgs(l,m,L,R); covcnt = cc;
	if(m < R) Query_sgs(m+1,r,L,R);
}
int main(){
	while(~scanf("%d%d%d",&n,&R,&C)){
		rep(i,1,R+1) h[i] = -1; e_cnt = 0;
		w.clear();
		rep(i,0,n){
			int x1,x2,y1,y2;
			scanf("%d%d%d%d",&x1,&y1,&x2,&y2);
			add(x1,y1,y2,1);
			add(x2+1,y1,y2,-1);
		}
		sort(all(w)); w.erase(unique(all(w)),w.end());
		Build(0,sz(w)-2);
		sg = 0;
		int po = 0;
		rep(i,1,R+1){
			for(int k=h[i];~k;k=ne[k]){
				e[k].l = lower_bound(all(w),e[k].l) - w.begin();
				e[k].r = lower_bound(all(w),e[k].r+1) - w.begin();
				Cover(0,sz(w)-2,e[k].l,e[k].r-1,e[k].v);
			}
			while(po < sz(w)-1 && w[po] < i) ++po;
			szcnt = 0 , nowx = i;
			covcnt = 0; if(po - 1 >= 0) Query_sz(0,sz(w)-2,0,po-1);
			covcnt = 0; if(po <= sz(w)-2) Query_sgs(0,sz(w)-2,po,sz(w)-2);
			if(szcnt&1) sg ^= i&(-i);
		}
		if(sg) puts("GTW wins"); else puts("DSY wins");
	}
	return 0;
}
\end{lstlisting}
\subsection{线段树合并}
\begin{lstlisting}[language=C++]
/*	@hiho-1193 线段树合并
    有根带权树各子树删除最少点变为树堆
    树堆->父节点权值大于儿子
    */
struct Index:vector<int>{
    void Pre(){
        sort(begin(),end());
        erase(unique(begin(),end()),end());
    }
    int rk(int x){
        return lower_bound(begin(),end(),x) - begin() + 1;
    }
}R;
int to[M] , ne[M] , h[N] , e;
void link(int u,int v){
    to[e] = v , ne[e] = h[u];
    h[u] = e++;
}
int n;
int val[N] , ans[N];
const int NODE = N * 60;
int lc[NODE] , rc[NODE] , sum[NODE] , L , rt[N] , up;
int newnode(){
    ++L; lc[L] = rc[L] = sum[L] = 0;
    return L;
}
int merge(int a,int b){
    if(a == 0) return b;
    if(b == 0) return a;
    int p = newnode();
    lc[p] = merge(lc[a],lc[b]);
    rc[p] = merge(rc[a],rc[b]);
    sum[p] = sum[a] + sum[b];
    return p;
}
int Query(int l,int r,int &p,int L,int R){
    if(L <= l && r <= R) return sum[p];
    if(p == 0) return 0;
    int m = l+r>>1;
    int ret = 0;
    if(L <= m ) ret += Query(l,m,lc[p],L,R);
    if(R > m ) ret += Query(m+1,r,rc[p],L,R);
    return ret;
}
void Update(int l,int r,int &p,int x,int d){
    if(p == 0) p = newnode();
    sum[p] += d;
    if(l == r) return;
    int m = l+r>>1;
    if(x <= m) Update(l,m,lc[p],x,d);
    else Update(m+1,r,rc[p],x,d);
}
int Query_R(int l,int r,int p,int x){
    if(p == 0 || sum[p] == 0) return 0;
    if(l == r) return l;
    int m = l+r>>1 , ret = 0;
    if(x <= m) ret = Query_R(l,m,lc[p],x);
    if(ret) return ret;
    return Query_R(m+1,r,rc[p],x);
}
void dfs(int c,int fa){
    rt[c] = newnode();
    for(int k=h[c];~k;k=ne[k]){
        if(to[k] == fa) continue;
        dfs(to[k] , c);
        rt[c] = merge(rt[c],rt[to[k]]);
    }
    ans[c] = Query(1,up,rt[c],1,val[c]) + 1;
    int l = val[c];
    Update(1,up,rt[c],val[c],1);
    int r = val[c] != up ? Query_R(1,up,rt[c],val[c]+1) : 0;
    if(r) Update(1,up,rt[c],r,-1);
}
int main(){
    while(~scanf("%d",&n)){
        R.clear();
        rep(i,0,n) scanf("%d",val+i) , R.push_back(val[i]);
        R.Pre(); rep(i,0,n) val[i] = R.rk(val[i]);
        up = R.size();
        fill(h,h+n,-1);e = 0;
        int u,v;
        rep(i,1,n){
            scanf("%d%d",&u,&v);
            link(u,v);link(v,u);
        }
        L = 0;dfs(0,-1);
        rep(i,0,n) printf("%d%c",ans[i]," \n"[i==n-1]);
    }
    return 0;
}
\end{lstlisting}
\subsection{遗失的支付宝密码}
\begin{lstlisting}[language=C++]
#include <algorithm>
#include <iostream>
#include <cstring>
#include <vector>
#include <cstdio>
#include <string>
#include <cmath>
#include <queue>
#include <set>
#include <map>
using namespace std;
typedef long long ll;
typedef double db;
typedef pair<int,int> pii;
typedef vector<int> vi;
#define de(x) cout << #x << "=" << x << endl
#define rep(i,a,b) for(int i=a;i<(b);++i)
#define all(x) (x).begin(),(x).end()
#define sz(x) (int)(x).size()
#define mp make_pair
#define pb push_back
#define fi first
#define se second
#define setIO(x) freopen(x".in","r",stdin);freopen(x".out","w",stdout)
const int N = 110;
typedef unsigned int ui;
int n , m;
map<vi,ui> Mp;
int vis[N],fa[N];
ui Pow[N],sum[N];
int F(int x){return fa[x]==x?x:fa[x]=F(fa[x]);}
vi gt(vi v,int c,int all){
    rep(i,0,all) fa[i] = i;
    rep(i,0,c) fa[F(v[i+c])] = F(v[i]);
    int cc = 0;
    rep(i,0,c+c) {
        if(!vis[F(v[i])]) vis[F(v[i])] = ++cc;
        v[i] = vis[F(v[i])] - 1;
    }
    rep(i,0,all) vis[i] = 0;
    return v;
}
ui dfs(int c,vi&v){
    if(Mp.find(v) != Mp.end()) return Mp[v];
    int cc = 0;
    for(int e : v) if(!vis[e]) vis[e] = true , ++cc;
    for(int e : v) vis[e] = false;
    ui ret = Pow[cc] * sum[n - c];
    if(c + 2 <= n){
        vi nv = v;
        for(int d=c+2;d<=n;d+=2){
            nv.pb(cc++);nv.pb(cc++);
            vi nvv = gt(nv,d/2,cc);
            ret -= dfs(d,nvv);
        }
    }
    return Mp[v] = ret;
}

int main(){
    cin >> n >> m;
    Mp.clear();
    Pow[0] = 1;rep(i,1,N) Pow[i] = Pow[i-1] * m;
    sum[0] = 1;rep(i,1,N) sum[i] = Pow[i] + sum[i-1];
    vi v;
    dfs(0,v);
    cout << Mp[v] - 1 << endl;
    return 0;
}
\end{lstlisting}
\subsection{链上第K大}
\begin{lstlisting}[language=C++]
/*	@hysbz-1146
	树链剖分链上第K大
*/
const int N = 80808 , M = N << 1;
int n,m;
int val[N] , h[N] , ne[M] , to[M] , e;
void link(int u,int v){
	to[e] = v , ne[e] = h[u];
	h[u] = e++;
}
int dep[N] , par[N][17] , lft[N] , rgt[N] , _;
void dfs(int c,int fa){
	par[c][0] = fa , dep[c] = dep[fa] + 1 ;
	lft[c] = ++_;
	for(int k=h[c];~k;k=ne[k])
		if(to[k] != fa)
			dfs(to[k] , c);
	rgt[c] = _;
}
int lca(int a,int b){
	if(dep[a] < dep[b]) swap(a,b);
	for(int i=16;i>=0;--i)
		if(dep[a] - dep[b] >>i&1)
			a = par[a][i];
	if(a == b) return a;
	for(int i=16;i>=0;--i)
		if(par[a][i] != par[b][i])
			a = par[a][i] , b = par[b][i];
	return par[a][0];
}
const int DOWN = -1 , UP = 1e8 + 1 , NODE = 12580000;
inline int get_id(int l,int r){
	return l+r|l!=r;
}
#define get int m=l+r>>1,id=get_id(l,r)
int rt[N<<1];
int lc[NODE] , rc[NODE] , sum[NODE] , L;
void Add(int l,int r,int &p,int x,int d){
	get;
	if(!p) p = ++L;
	if(l == r) sum[p] += d;
	else{
		if(x <= m) Add(l,m,lc[p],x,d);
		else Add(m+1,r,rc[p],x,d);
		sum[p] = sum[lc[p]] + sum[rc[p]];
	}
}
void add(int l,int r,int L,int R,int val,int d){
	get;
	if(L <= l && r <= R) Add(DOWN,UP,rt[id],val,d);
	else{
		if(L <= m) add(l,m,L,R,val,d);
		if(R > m) add(m+1,r,L,R,val,d);
	}
}
vi pl , ml;
void get_rts(int l,int r,int x,int d){
	get;
	if(d == 1) pl.push_back(rt[id]);
	else ml.push_back(rt[id]);
	if(l == r) return;
	else{
		if(x <= m) get_rts(l,m,x,d);
		else get_rts(m+1,r,x,d);
	}
}
int Query_kth(int l,int r,int K){
	get;
	if(l == r) return l;
	int cnt = 0;
	rep(i,0,sz(pl)) cnt += sum[rc[pl[i]]];
	rep(i,0,sz(ml)) cnt -= sum[rc[ml[i]]];
	if(cnt >= K){
		rep(i,0,sz(pl)) pl[i] = rc[pl[i]];
		rep(i,0,sz(ml)) ml[i] = rc[ml[i]];
		return Query_kth(m+1,r,K);
	}
	else{
		rep(i,0,sz(pl)) pl[i] = lc[pl[i]];
		rep(i,0,sz(ml)) ml[i] = lc[ml[i]];
		return Query_kth(l,m,K - cnt);
	}
}
int Query(int a,int b,int K){
	pl.clear();ml.clear();
	int c = lca(a,b);
	get_rts(1,n,lft[a],1);
	get_rts(1,n,lft[b],1);
	get_rts(1,n,lft[c],-1);
	if(par[c][0]) get_rts(1,n,lft[par[c][0]],-1);
	return Query_kth(DOWN,UP,K);
}
int main(){
	scanf("%d%d",&n,&m);
	rep(i,1,n+1) scanf("%d",val+i);
	e = 0;fill(h+1,h+1+n,-1);
	int u , v;
	rep(i,1,n){
		scanf("%d%d",&u,&v);
		link(u,v);link(v,u);
	}
	_ = 0;dfs(1,0);
	for(int j=1;1<<j<=n;++j)
		for(int i=1;i<=n;++i)
			par[i][j] = par[par[i][j-1]][j-1];
	rep(i,1,n+1) add(1,n,lft[i],rgt[i],val[i],1);
	int k , a , b;
	rep(i,0,m){
		scanf("%d%d%d",&k,&a,&b);
		if(k == 0){
			add(1,n,lft[a],rgt[a],val[a],-1);
			add(1,n,lft[a],rgt[a],val[a] = b,1);
		}
		else{
			int ans = Query(a,b,k);
			if(ans == -1) puts("invalid request!");
			else printf("%d\n",ans);
		}
	}
	return 0;
}
\end{lstlisting}
\subsection{青云的机房组网方案}
\begin{lstlisting}[language=C++]
#include <algorithm>
#include <iostream>
#include <cstring>
#include <vector>
#include <cstdio>
#include <string>
#include <queue>
#include <set>
#include <map>
#include <bitset>
using namespace std;
typedef long long ll;
typedef double db;
typedef pair<int,int> pii;
typedef vector<int> vi;
#define de(x) cout << #x << "=" << x << endl
#define rep(i,a,b) for(int i=a;i<b;++i)
#define all(x) (x).begin(),(x).end()
#define sz(x) (int)(x).size()
#define mp make_pair
#define pb push_back
#define fi first
#define se second
#define setIO(x) freopen(x".in","r",stdin);freopen(x".out","w",stdout);
const int N = 101000;
vi G[N] , fac[N];
int mu[N] , p[N] , dep[N] , lg[N<<1] , lft[N<<1] , n;
ll ans;
struct RMQ{
    int a[20][N<<1] , L;
    int rmin(int x,int y){return dep[x] < dep[y] ? x : y;}
    void Build(){
        rep(i,1,20){
            int lim = L + 1 - (1<<i);
            rep(j,0,lim) a[i][j] = rmin(a[i-1][j] , a[i-1][j+(1<<i>>1)]);
        }
    }
    int lca(int x,int y){
        if(x == y) return x;
        x = lft[x] , y = lft[y];
        if(x > y) swap(x , y);
        int j = lg[y-x+1];
        return rmin(a[j][x] , a[j][y+1-(1<<j)]);
    }
    void add(int x){
        a[0][L++] = x;
    }
}R;
vi V[N];
void dfs(int c,int fa){
    dep[c] = dep[fa] + 1;
    lft[c] = R.L;R.add(c);
    for(int t : fac[p[c]]) V[t].pb(c);
    for(int t : G[c]) if(t != fa){
        dfs(t , c);
        R.add(c);
    }
}
vi g[N];int tp[N] , _ , w;
ll dn[N] , up[N] , sz[N] , rt;
void DFS1(int c,int fa){
    dn[c] = 0;sz[c] = p[c] % w == 0;
    for(int t : g[c]) if(t != fa){
        DFS1(t , c);
        dn[c] += sz[t] * (dep[t] - dep[c]) + dn[t];
        sz[c] += sz[t];
    }
}
void DFS2(int c,int fa){
    if(fa == 0) up[c] = 0;
    for(int t : g[c]) if(t != fa){
        up[t] = (sz[rt] - 2 * sz[t]) * (dep[t] - dep[c]) + up[c] + dn[c] - dn[t];
        DFS2(t , c);
    }
}
ll solve(vi&v){
    if(sz(v) <= 1) return 0;
    _ = 0;
    vi del;del.pb(tp[_++] = v[0]);
    rep(i,1,sz(v)){
        int lca = R.lca(tp[_-1] , v[i]);
        vi l;while(_ > 0 && dep[lca] < dep[tp[_-1]]) l.pb(tp[--_]);
        if(_ == 0 || lca != tp[_-1]) del.pb(tp[_++] = lca);
        l.pb(tp[_-1]);del.pb(tp[_++] = v[i]);
        rep(i,1,sz(l)) g[l[i]].pb(l[i-1]);
    }
    rep(i,0,_-1) g[tp[i]].pb(tp[i+1]);
    DFS1(rt = tp[0] , 0);
    DFS2(rt , 0);
    ll ret = 0;
    for(int t : del) {
        if(p[t] % w == 0) ret += dn[t] + up[t];
        g[t].clear();
    }
    return ret / 2;
}

int main(){
    rep(i,2,N+N)  lg[i] = lg[i-1] + (i == (i&-i));
    mu[1] = 1;rep(i,1,N) for(int j=i+i;j<N;j+=i) mu[j] -= mu[i] , fac[j].pb(i);
    rep(i,1,N) fac[i].pb(i);
    scanf("%d",&n);
    rep(i,1,n+1) scanf("%d",p + i);
    rep(i,1,n){
        int u , v;
        scanf("%d%d",&u,&v);
        G[u].pb(v);
        G[v].pb(u);
    }
    dfs(1 , 0);
    R.Build();
    rep(i,1,N) if(mu[i]) ans += mu[i] * solve(V[w = i]);
    printf("%lld\n",ans);
    return 0;
}
\end{lstlisting}
